% \iffalse
%<*internal>
% \fi
\ifcsname ifThisIsTheMainRun\endcsname
  \relax
\else
  \expandafter\newif\csname ifThisIsTheMainRun\endcsname
\fi
% \iffalse
%</internal>
%<*internal|class|theme|theme2|theme3|theme4|palette|mscwg|msdwg|mig>
\def\Version{2025/03/07 v5}
%</internal|class|theme|theme2|theme3|theme4|palette|mscwg|msdwg|mig>
%<*internal>
\iffalse
%</internal>
%<*class|theme|theme2|theme3|theme4|palette|mscwg|msdwg|mig>
\NeedsTeXFormat{LaTeX2e}[1999/12/01]
%</class|theme|theme2|theme3|theme4|palette|mscwg|msdwg|mig>
%<*class>
\ProvidesClass{rdaslides}
    [\Version\space Class for Research Data Alliance presentations]
%</class>
%<*theme>
\ProvidesPackage{beamerthemeRDA}
    [\Version\space Beamer theme for Research Data Alliance presentations (2013 version)]
%</theme>
%<*theme2>
\ProvidesPackage{beamerthemeRDA2016}
    [\Version\space Beamer theme for Research Data Alliance presentations (2016 version)]
%</theme2>
%<*theme3>
\ProvidesPackage{beamerthemeRDA2020}
    [\Version\space Beamer theme for Research Data Alliance presentations (2020 version)]
%</theme3>
%<*theme4>
\ProvidesPackage{beamerthemeRDA2024}
    [\Version\space Beamer theme for Research Data Alliance presentations (2024 version)]
%</theme4>
%<*palette>
\ProvidesPackage{rdacolors}
    [\Version\space Colour palette for Research Data Alliance documents]
%</palette>
%<*mscwg>
\ProvidesPackage{rdamscwg}
    [\Version\space Logo and details of the RDA Metadata Standards Catalog Working Group]
%</mscwg>
%<*msdwg>
\ProvidesPackage{rdamsdwg}
    [\Version\space Logo and details of the RDA Metadata Standards Directory Working Group]
%</msdwg>
%<*mig>
\ProvidesPackage{rdamig}
[\Version\space Logo and details of the RDA Metadata Interest Group]
%</mig>
%<*sample1>
\documentclass{rdaslides}
%</sample1>
%<*sample2>
\PassOptionsToClass{aspectratio=169}{beamer}
\documentclass[2016]{rdaslides}
%</sample2>
%<*sample3>
\PassOptionsToClass{aspectratio=169}{beamer}
\documentclass[2020]{rdaslides}
%</sample3>
%<*sample4>
\PassOptionsToClass{aspectratio=169}{beamer}
\documentclass[2024]{rdaslides}
%</sample4>
%<*sample1|sample2|sample3|sample4>
\usepackage[scaled=0.95,tabular]{sourceserifpro}
\usepackage[scaled=0.95,tabular]{sourcesanspro}
\usepackage[varl,varqu]{zi4}
\title{Sample Presentation}
\author{Anne Author}
\date{2016-07-01}
\begin{document}
  \begin{frame}{Sample slide}
    Here is a list of things to think about
    \begin{itemize}
      \item What problem are you trying to solve?
      \item How do you propose solve it?
      \item Can you do it in 18 months?
      \item Who will adopt your solution?
      \begin{itemize}
        \item Industry
        \item Academia
        \begin{itemize}
          \item Domain scientists
          \item Infrastructure providers
        \end{itemize}
      \end{itemize}
      \item How will you sustain it?
    \end{itemize}
  \end{frame}
\end{document}
%</sample1|sample2|sample3|sample4>
%<*readme>
The rdaslides class: Research Data Alliance presentations
=========================================================

The rdaslides LaTeX class is intended to produce slides for Research
Data Alliance (RDA) presentations, or an accompanying transcript, or both.
It is based on the [beamerswitch] class.

Internally, rdaslides uses one of three presentation themes, called ‘RDA’,
‘RDA2016’ and ‘RDA2020’ respectively, which can be used independently
within beamer.

Installation
------------

### Pre-requisites ###

The documentation uses fonts from the XCharter and sourcesanspro
packages, as well as sourcecodepro if XeLaTeX or LuaLaTeX is used,
or zi4 (inconsolata) otherwise. To compile the documentation
successfully, you will need the minted package installed and working.

### Automated way ###

A makefile is provided which you can use with the Make utility:

  * Running `make rdaslides.cls` generates the derived files:

      - README.md
      - rdaslides.cls
      - rdacolors.sty
      - beamerthemeRDA.sty
      - beamerthemeRDA2016.sty
      - beamerthemeRDA2020.sty
      - beamerthemeRDA2024.sty
      - rdamig.sty
      - rdamscwg.sty
      - rdamsdwg.sty
      - rdaslides.ins
      - rdaslides-sample-RDA.tex
      - rdaslides-sample-RDA2016.tex
      - rdaslides-sample-RDA2020.tex
      - rdaslides-sample-RDA2024.tex

  * Running `make` generates the above plus

      - rda-logo.eps
      - rdaslides.pdf
      - rdaslides-slides.pdf
      - rdaslides-sample-RDA.pdf
      - rdaslides-sample-RDA2016.pdf
      - rdaslides-sample-RDA2020.pdf
      - rdaslides-sample-RDA2024.pdf

  * Running `make inst` installs the files (and images) in the user's
    TeX tree.  (To undo, run `make uninst`.)
  * Running `make install` installs the files (and images) in the
    local TeX tree. (To undo, run `make uninstall`.)

The makefile is set up to use latexmk and lualatex by default.
If this causes difficulty you could change it to use pdflatex directly
instead.

### Manual way ###

To install the class from scratch, follow these instructions. If you have
downloaded the zip file from the [Releases] page on GitHub, you can skip the
first five steps.

 1. Run `etex rdaslides.dtx` to generate the class and package files.

 2. Compile rdaslides-sample-RDA.tex and rdaslides-sample-RDA2016.tex to get
    the example figures used in the documentation.

 3. If you intend to use the theme in DVI mode, you will need to convert the
    logo to EPS format; here is one way to do it:

    ~~~{.bash}
    pdftops -f 1 -l 1 -eps rda-logo.pdf rda-logo.eps
    ~~~

    If you intend to compile the documentation in DVI mode, you will also need
    to transform the output of step 2 to EPS using, say, `dvips` or `pdftops`,
    depending on how you did it.

 4. Compile rdaslides.dtx using your favourite version of LaTeX with shell
    escape enabled (as required by minted for typesetting the listings). You
    will also need to run it through `makeindex`. This will generate the main
    documentation (DVI or PDF).

 5. Compile rdaslides.dtx a second time with `-jobname=rdaslides-slides`
    as a command line option to generate the sample slides. Again, you will
    need to enable shell escape so that minted can mark up the code listings.

 6. Move the files to your TeX tree as follows:

      - `source/latex/rdaslides`:
        rdaslides.dtx,
        rdaslides.ins
      - `tex/generic/logos-rda`:
        rda-logo.eps,
        rda-logo.pdf
      - `tex/latex/rdaslides`:
        rdaslides.cls,
        rdacolors.sty,
        beamerthemeRDA.sty,
        beamerthemeRDA2016.sty,
        beamerthemeRDA2020.sty,
        beamerthemeRDA2024.sty,
        rdamig.sty,
        rdamscwg.sty,
        rdamsdwg.sty,
      - `tex/latex/rdaslides/img2013`:
        rda-bg-normal.jpeg,
        rda-bg-title1.jpeg,
        rda-bg-title2.jpeg
      - `tex/latex/rdaslides/img2020`:
        rda-bg-watermark.jpeg
        rda-link-white.png
        rda-twitter-white.png
      - `tex/latex/rdaslides/img2024`:
        rda24-*.png
      - `doc/latex/rdaslides`:
        rdaslides.pdf,
        rdaslides-sample-RDA.tex,
        rdaslides-sample-RDA.pdf,
        rdaslides-sample-RDA2016.tex,
        rdaslides-sample-RDA2016.pdf,
        rdaslides-sample-RDA2020.tex,
        rdaslides-sample-RDA2020.pdf,
        rdaslides-sample-RDA2024.tex,
        rdaslides-sample-RDA2024.pdf,
        rdaslides-slides.pdf,
        README.md

 7. You may then have to update your installation's file name database
    before TeX and friends can see the files.

Licence
-------

Copyright 2016 Alex Ball.

This work consists of various image files, the documented LaTeX file
rdaslides.dtx, and a Makefile.

The text files contained in this work may be distributed and/or modified
under the conditions of the [LaTeX Project Public License (LPPL)][lppl],
either version 1.3c of this license or (at your option) any later
version.

The rights to the image files distributed with this bundle (except
rda-twitter-white.png) are held by the [Research Data Alliance][rda].
Usage guidelines may be found with the [RDA Communication Kit][rda-kit].

This work is "maintained" (as per LPPL maintenance status) by
[Alex Ball][me].

[beamerswitch]: https://github.com/alex-ball/beamerswitch
[Releases]: https://github.com/alex-ball/rdaslides/releases
[lppl]: http://www.latex-project.org/lppl.txt
[rda]: https://rd-alliance.org/
[rda-kit]: https://www.rd-alliance.org/communication-kit/
[me]: http://alexball.me.uk/

%</readme>
%<*internal>
\fi
\def\nameofplainTeX{plain}
\ifx\fmtname\nameofplainTeX
  \ThisIsTheMainRuntrue
\else
  \expandafter\begingroup
  \makeatletter
  \protected@edef\su@SubString{-}%
  \protected@edef\su@String{\jobname}%
  \def\su@compare#1-#2\@nil{%
    \def\su@param{#2}%
    \ifx\su@param\@empty
      \expandafter\@secondoftwo
    \else
      \expandafter\@firstoftwo
    \fi
  }%
  \def\su@comp@compare#1#2{\su@compare#2\@nnil#1\@nil}%
  \expandafter\expandafter\expandafter\su@comp@compare%
  \expandafter\expandafter\expandafter{%
  \expandafter\su@SubString\expandafter
  }\expandafter{\su@String}%
  {\ThisIsTheMainRunfalse}{\ThisIsTheMainRuntrue}
  \makeatother
\fi
\ifThisIsTheMainRun
%</internal>
%<*install>
\input docstrip.tex
\keepsilent
\askforoverwritefalse
\preamble

----------------------------------------------------------------
The rdaslides class: Research Data Alliance presentations
Author:  Alex Ball
E-mail:  a.j.ball@bath.ac.uk
License: Released under the LaTeX Project Public License v1.3c or later
See:     http://www.latex-project.org/lppl.txt
----------------------------------------------------------------

\endpreamble
\postamble

Copyright (C) 2016 Alex Ball <a.j.ball@bath.ac.uk>
\endpostamble

\usedir{tex/latex/rdaslides}
\generate{
  \file{rdaslides.cls}{\from{\jobname.dtx}{class}}
  \file{rdacolors.sty}{\from{\jobname.dtx}{palette}}
  \file{beamerthemeRDA.sty}{\from{\jobname.dtx}{theme}}
  \file{beamerthemeRDA2016.sty}{\from{\jobname.dtx}{theme2}}
  \file{beamerthemeRDA2020.sty}{\from{\jobname.dtx}{theme3}}
  \file{beamerthemeRDA2024.sty}{\from{\jobname.dtx}{theme4}}
  \file{rdamig.sty}{\from{\jobname.dtx}{mig}}
  \file{rdamsdwg.sty}{\from{\jobname.dtx}{msdwg}}
  \file{rdamscwg.sty}{\from{\jobname.dtx}{mscwg}}
}
\usedir{doc/latex/rdaslides}
\generate{
  \file{rdaslides-sample-RDA.tex}{\from{\jobname.dtx}{sample1}}
  \file{rdaslides-sample-RDA2016.tex}{\from{\jobname.dtx}{sample2}}
  \file{rdaslides-sample-RDA2020.tex}{\from{\jobname.dtx}{sample3}}
  \file{rdaslides-sample-RDA2024.tex}{\from{\jobname.dtx}{sample4}}
}
%</install>
%<install>\endbatchfile
%<*internal>
\usedir{source/latex/rdaslides}
\generate{
  \file{\jobname.ins}{\from{\jobname.dtx}{install}}
}
\nopreamble\nopostamble
\usedir{doc/latex/rdaslides}
\generate{
  \file{README.md}{\from{\jobname.dtx}{readme}}
}
\fi
\ifx\fmtname\nameofplainTeX
  \expandafter\endbatchfile
\else
  \expandafter\endgroup
\fi

%</internal>
%<*driver>
\ProvidesFile{rdaslides.dtx}
    [\Version\space Class for Research Data Alliance presentations]
\PassOptionsToClass{aspectratio=169}{beamer}
\PassOptionsToPackage{highlightmode=immediate}{minted}
\documentclass[article,2020]{rdaslides}
\let\RdaslidesMaketitle=\maketitle

% Basic typography
\usepackage[scaled=0.95,tabular]{sourceserifpro}
\usepackage[scaled=0.95,tabular]{sourcesanspro}
\usepackage[varl,varqu]{zi4}
\usepackage{iftex}
\ifPDFTeX
  \usepackage[varl,varqu]{zi4}
\else
  \usepackage{fontspec}
  \setmonofont{Source Code Pro}%
    [Scale=MatchLowercase
    ,RawFeature={extend=0.83}
    ,BoldFont={Source Code Pro Bold}
    ,ItalicFont={Source Code Pro Italic}
    ,BoldItalicFont={Source Code Pro Bold Italic}
    ,ItalicFeatures={FakeSlant=0.03}
    ,BoldItalicFeatures={FakeSlant=0.03}
    ]%
\fi
\usepackage{metalogo}
\usepackage{multicol}

% For typesetting the documentation generally
\usepackage{tcolorbox}
\tcbuselibrary{documentation,breakable,minted,skins}
\mode<article>{
  \usepackage{doc}
}
\colorlet{Option}{violet}
\colorlet{Command}{red!75!black}
\colorlet{Environment}{blue!75!black}
\colorlet{Value}{olive!75!black}
\colorlet{Color}{cyan!75!black}
\colorlet{ExampleFrame}{rdagrey}
\colorlet{ExampleBack}{rdayellow!10}
\tcbset
  { colframe=rdagrey
  , colback=rdayellow!10
  , left=2mm
  , right=2mm
  , listing engine=minted
  , minted options=
    { breaklines
    , fontsize=\footnotesize
    %, linenos
    %, numbersep=20pt
    %, firstnumber=last
    , stripnl
    }
  , index format=pgf
  , color command=Command
  , color environment=Environment
  , color key=Option
  , color value=Value
  , color color=Color
  , docexample/.style=
    { colframe=ExampleFrame
    , colback=ExampleBack
    , before skip=1em plus 0.2em minus 0.2em
    , after skip=1em plus 0.2em minus 0.2em
    , fontlower=\footnotesize
    , skin=enhanced
    }
  }
\renewcommand{\theFancyVerbLine}{\footnotesize\itshape\color{gray}\arabic{FancyVerbLine}}

% For typesetting the user documentation
\newcommand{\pkg}[1]{\href{http://www.ctan.org/pkg/#1}{\textsf{#1}}}
\newrobustcmd{\bthm}[1]{\textcolor{Color}{\textsf{#1}}}
\let\tcbcs=\cs
\renewcommand*{\cs}[1]{\textcolor{Command}{\tcbcs{#1}}}
\def\sqbrackets#1{%
  \texttt{\textcolor{Option}{[}#1\textcolor{Option}{]}}}
\def\brackets#1{%
  \texttt{\textcolor{Environment}{\char`\{}#1\textcolor{Environment}{\char`\}}}}
\def\marg#1{%
  \textcolor{Environment}{\ttfamily\char`\{}\meta{#1}\textcolor{Environment}{\ttfamily\char`\}}}
\newcommand*{\env}[1]{\textcolor{Environment}{\ttfamily #1}}
\newcommand*{\key}[1]{\textcolor{Option}{\ttfamily #1}}
\newcommand*{\val}[1]{\textcolor{Value}{\ttfamily #1}}
\newcommand{\sample}[2][1em]{\bgroup\setlength{\fboxsep}{0pt}\fbox{\color{#2}\rule{#1}{#1}}\egroup}

% For typesetting the implementation
\mode<article>{
  \renewenvironment{macro}[1]{%
    \def\MyName{#1}%
    \index{\MyName@\tcbIndexPrintComC {\MyName}|(emph}%
  }{%
    \ifdef{\MyName}{}{%
      \errmessage{You have closed a macro environment you have not opened on \the\inputlineno.}%
    }
    \index{\MyName@\tcbIndexPrintComC {\MyName}|)}%
  }
  \renewenvironment{environment}[1]{%
    \def\MyName{#1}%
    \index{\MyName@\tcbIndexPrintEnvCA {\MyName}|(emph}%
    \index{Environments!\MyName@\tcbIndexPrintEnvC {\MyName}|(emph}%
  }{%
    \ifdef{\MyName}{}{%
      \errmessage{You have closed an environment environment you have not opened on \the\inputlineno.}%
    }
    \index{Environments!\MyName@\tcbIndexPrintEnvC {\MyName}|)}%
    \index{\MyName@\tcbIndexPrintEnvCA {\MyName}|)}%
  }
}
\newenvironment{optionkey}[1]{%
  \def\MyName{#1}%
  \index{\MyName@\tcbIndexPrintKeyCA {\MyName}|(emph}%
  \index{Keys!\MyName@\tcbIndexPrintKeyC {\MyName}|(emph}%
}{%
  \ifdef{\MyName}{}{%
    \errmessage{You have closed an optionkey environment you have not opened on \the\inputlineno.}%
  }
  \index{Keys!\MyName@\tcbIndexPrintKeyC {\MyName}|)}%
  \index{\MyName@\tcbIndexPrintKeyCA {\MyName}|)}%
}
\newenvironment{optionvalue}[1]{%
  \def\MyName{#1}%
  \index{\MyName@\tcbIndexPrintValCA {\MyName}|(emph}%
  \index{Values!\MyName@\tcbIndexPrintValC {\MyName}|(emph}%
}{%
  \ifdef{\MyName}{}{%
    \errmessage{You have closed an optionvalue environment you have not opened on \the\inputlineno.}%
  }
  \index{Values!\MyName@\tcbIndexPrintValC {\MyName}|)}%
  \index{\MyName@\tcbIndexPrintValCA {\MyName}|)}%
}
\makeatletter
\newcommand{\resetmintedformat}{%
  % Comments
  \expandafter\def\csname PYGdefault@tok@c\endcsname{\let\PYGdefault@it=\textit\def\PYGdefault@tc####1{\textcolor{gray}{####1}}}
  % Command sequences
  \expandafter\def\csname PYGdefault@tok@k\endcsname{\def\PYGdefault@tc####1{\textcolor{Command}{####1}}}
  % Optional arguments
  \expandafter\def\csname PYGdefault@tok@na\endcsname{\def\PYGdefault@tc####1{\textcolor{Option}{####1}}}
  % Braces
  \expandafter\def\csname PYGdefault@tok@nb\endcsname{\def\PYGdefault@tc####1{\textcolor{Environment}{####1}}}
}
\apptocmd{\minted@checkstyle}{\resetmintedformat}{}{}
\makeatother
\mode<article>{
  \MakeShortVerb{\|}
}
\makeatletter
\let\PrintMacroName\@gobble
\let\PrintEnvName\@gobble
\renewenvironment{tcb@manual@entry}{%
  \begin{list}{}{%
    \setlength{\topsep}{0pt}
    \setlength{\partopsep}{0pt}
    \setlength{\leftmargin}{\kvtcb@doc@left}%
    \setlength{\itemindent}{0pt}%
    \setlength{\itemsep}{0pt}%
    \setlength{\parsep}{0pt}%
    \setlength{\rightmargin}{\kvtcb@doc@right}%
  }\item
}{\end{list}}
\makeatother
% This bit inspired by ydoc
\mode<article>
\makeatletter
\newwrite\ydocwrite
\def\ydocfname{\jobname.tcbtemp}
\def\ydoc@catcodes{%
  \let\do\@makeother
  \dospecials
  \catcode`\\=\active
  \catcode`\^^M=\active
  \catcode`\ =\active
}
\def\macrocode{%
  \begingroup
  \ydoc@catcodes
  \macro@code
}
\def\endmacrocode{}
\begingroup
\endlinechar\m@ne
\@firstofone{%
  \catcode`\|=0\relax
  \catcode`\(=1\relax
  \catcode`\)=2\relax
  \catcode`\*=14\relax
  \catcode`\{=12\relax
  \catcode`\}=12\relax
  \catcode`\ =12\relax
  \catcode`\%=12\relax
  \catcode`\\=\active
  \catcode`\^^M=\active
  \catcode`\ =\active
}*
|gdef|macro@code#1^^M%    \end{macrocode}(*
|endgroup|expandafter|macro@@code|expandafter(|ydoc@removeline#1|noexpand|lastlinemacro)*
)*
|gdef|ydoc@removeline#1^^M(|noexpand|firstlinemacro)*
|gdef|ydoc@defspecialmacros(*
|def^^M(|noexpand|newlinemacro)*
|def (|noexpand|spacemacro)*
|def\(|noexpand|bslashmacro)*
)*
|gdef|ydoc@defrevspecialmacros(*
|def|newlinemacro(|noexpand^^M)*
|def|spacemacro(|noexpand )*
|def|bslashmacro(|noexpand\)*
)*
|endgroup
\def\macro@@code#1{%
  {\ydoc@defspecialmacros
    \xdef\themacrocode{#1}}%
  \PrintMacroCode
  \end{macrocode}%
}
\def\PrintMacroCode{%
  \begingroup
  \let\firstlinemacro\empty
  \let\lastlinemacro\empty
  \def\newlinemacro{^^J}%
  \let\bslashmacro\bslash
  \let\spacemacro\space
  \immediate\openout\ydocwrite=\ydocfname\relax
  \immediate\write\ydocwrite{\themacrocode}%
  \immediate\closeout\ydocwrite
  \let\input\@input
  \tcbinputlisting{breakable,listing only,docexample,listing file=\ydocfname}%
  \endgroup
}
\makeatother

\DisableCrossrefs
\makeindex
%\CodelineIndex
\RecordChanges
\mode
<all>

\let\maketitle=\RdaslidesMaketitle

%\usepackage{rdamscwg}

%\def\licenseLogo{\includegraphics[width=\hsize]{cc_by}}
%\def\licenseStatement{Except where otherwise stated,%
% this work is licensed under the Creative Commons Attribution 4.0 International licence}
%\def\licenseUrl{https://creativecommons.org/licenses/by/4.0/}

\mode<presentation>{
  \usepackage{readprov}
}
\GetFileInfo{rdaslides.cls}
\title{The \protect\textsf{rdaslides} class: Research Data Alliance presentations}
\author{Alex Ball}
\authorurl{http://alexball.me.uk/}
\institute{University of Bath}
\StrSubstitute{\filedate}{/}{-}[\IsoFileDate]
\date{\IsoFileDate}
\occasion{An RDA Plenary}
%\hashtag{dummy}

\begin{document}
\begin{frame}
\maketitle
\end{frame}

\begin{absquote}
This is the documentation for and a demonstration of \filename, ‘\fileinfo’, \fileversion, dated \printdateTeX{\filedate}.
\end{absquote}

\section{Usage}

\subsection{Loading the class}

The class is loaded in the usual way.

\begin{frame}[fragile]{Using the class}
\begin{tcolorbox}[docexample,fontupper=\small]
\cs{documentclass}\oarg{options}\brackets{rdaslides}
\end{tcolorbox}

The class defaults to producing slides. You can change this with an option:
\begin{docKey}{handout}{}{no value, initially unset}
Lay slides out two to an A4 page, for easy printing.
\end{docKey}
\begin{docKey}{trans}{}{no value, initially unset}
Produce less dynamic slides: useful for archival versions.
\only<article>{See the \pkg{beamer} manual for a more detailed explanation
  of how this ‘transparencies version’ behaves.}
\end{docKey}
\begin{docKey}{article}{}{no value, initially unset}
Produce a document in article mode: useful for transcripts.
\end{docKey}
\begin{docKey}{set}{}{no value, initially unset}
Produce a document in article mode, plus a set of slides\only<article>{
(suffixed with ‘\texttt{-slides}’)}.
\end{docKey}
\end{frame}

Since this class was first released, the mode switching aspects have been
improved and separated out into a separate class file, \pkg{beamerswitch}.
The options above are only a subset; please see the \pkg{beamerswitch}
documentation for the full set. Note that the \key{set} option is now simply
shorthand for \key{article} and \key{alsobeamer}.

% A 4:3 slide has a line width of 108mm.
% A 16:9 slide has a line width of 140mm.
% The article page has a line width of 150mm.
\begin{frame}[fragile]{Using the class}
\setlength{\fboxsep}{0pt}%
\only<presentation>{\setlength{\columnsep}{7mm}}%
\only<article>{\setlength{\columnsep}{12mm}}%
\begin{multicols}{2}[There are also options for choosing between the four available themes.]
\begin{docKey}{2013}{}{initially set}
  Use the \bthm{RDA} theme.
\end{docKey}
\par\only<presentation>{\medskip}%
\par\bgroup\centering
\fbox{\includegraphics[height=3.6cm]{rdaslides-sample-RDA}}
\par\egroup
\begin{docKey}{2016}{}{initially unset}
\raggedright
  Use the \bthm{RDA2016} theme.
\end{docKey}
\par\only<presentation>{\medskip}%
\par\bgroup\centering
\fbox{\includegraphics[height=3.6cm]{rdaslides-sample-RDA2016}}
\par\egroup
\columnbreak
\begin{docKey}{2020}{}{initially unset}
\raggedright
  Use the \bthm{RDA2020} theme.
\end{docKey}
\par\only<presentation>{\medskip}%
\par\bgroup\centering
\fbox{\includegraphics[height=3.6cm]{rdaslides-sample-RDA2020}}
\par\egroup
\begin{docKey}{2024}{}{initially unset}
\raggedright
  Use the \bthm{RDA2024} theme.
\end{docKey}
\par\only<presentation>{\medskip}%
\par\bgroup\centering
\fbox{\includegraphics[height=3.6cm]{rdaslides-sample-RDA2024}}
\par\egroup
\end{multicols}
\end{frame}

\subsection{Loading the theme}

\begin{frame}[fragile]{Using the beamer theme}
\only<presentation>{\setlength{\columnsep}{7mm}}%
\only<article>{\setlength{\columnsep}{5mm}}%
If you don't want the article mode settings introduced by the full class,
you can simply load the theme directly into \pkg{beamer}:

\begin{multicols}{2}
\begin{dispListing}
\documentclass{beamer}
\usetheme{RDA}
\end{dispListing}
\begin{dispListing}
\documentclass{beamer}
\usetheme{RDA2016}
\end{dispListing}
\columnbreak
\begin{dispListing}
\documentclass{beamer}
\usetheme{RDA2020}
\end{dispListing}
\begin{dispListing}
\documentclass{beamer}
\usetheme{RDA2024}
\end{dispListing}
\end{multicols}
\end{frame}

\subsection{Preamble metadata}

The class file provides some additional commands for describing your presentation.
\begin{frame}{Metadata}
When filling out the document metadata, you can use the regular \pkg{beamer} conventions with a few differences:

\begin{docCommand}{date}{\marg{ISO date}}
\only<article>{This is not a new command (!) but it is handled slightly differently.}
Enter the date in ISO format, e.g.\ 2011-12-13.
\end{docCommand}
\begin{docCommand}{occasion}{\marg{event name}}
Use this to provide the name of the event where the presentation will be given.
\end{docCommand}
\begin{docCommand}{authorurl}{\marg{URL}}
Use this to provide a URL where more details about the author may be consulted (not so useful for multi-author presentations).
\end{docCommand}
\end{frame}

\begin{frame}{Social media}
\begin{docCommand}{handles}{\marg{comma-separated list}}
Use this to provide Twitter handles after the official RDA one. Omit the initial @ sign, as the class will add it for you. Defaults to the handles for RDA Europe and RDA US. Remember to escape any underscores.
\end{docCommand}
\begin{docCommand}{hashtag}{\marg{hashtag}}
Use this to specify exactly one hashtag for sharing on social media. Omit the hash (\#), as the class will add it for you.
\end{docCommand}
\end{frame}

The above changes are not present in the standalone \pkg{beamer} theme.
You can however sneak the information into your presentation by defining
\cs{insertoccasion}, \cs{insertauthorurl} and \cs{inserthashtag} respectively.
To trigger the change in date handling, load the \pkg{isodate} package.

\subsection{Customizing the title slide}

You can add elements to the title slide by using the hooks provided.
(The hooks are also recognized in article mode but they mainly affect the
\cs{finale} command, of which more later.) You use the hooks by defining them,
so to use \cs{rdaGroupName}, for example, you would need to do something like this:
\begin{dispListing}
\def\rdaGroupName{Metadata}
\end{dispListing}

You can associate your presentation with a particular group by defining the following hooks.

\begin{frame}[fragile]{Define-It-Yourself Hooks (RDA Group)}
\begin{docCommand}{rdaGroupLogo}{}
This should insert the logo of an RDA group. For best results, scale it to fit the width of the bounding box and centre it vertically, like so:
\begin{dispListing}
\def\rdaGroupLogo{\adjustimage{width=\hsize,valign=m}{filename}}
\end{dispListing}
\end{docCommand}
\begin{docCommand}{rdaGroupName}{}
The name of the RDA group, e.g. ‘Metadata’.
\end{docCommand}
\begin{docCommand}{rdaGroupType}{}
The type of RDA group, e.g. ‘Interest Group’.
\end{docCommand}
\begin{docCommand}{rdaGroupUrl}{}
The URL of the group web page.
\end{docCommand}
\end{frame}

By way of demonstration, the package \textsf{rdamscwg.sty} is provided for
associating presentations with the Metadata Standards Catalog Working Group.
Use it in the usual way:

\begin{dispListing}
\usepackage{rdamscwg}
\end{dispListing}

Contributions of similar packages representing other groups are welcome,
but there is no intention to provide a comprehensive set.

If releasing the presentation under licence, you can declare this by defining
the following hooks.

\begin{frame}[fragile]{Define-It-Yourself Hooks (Licence)}
\begin{docCommand}{licenseLogo}{}
This should insert the logo of the licence under which the presentation is released, if applicable. For best results, scale it to fit the width of the bounding box like so:
\begin{dispListing}
\def\licenseLogo{\insertgraphics[width=\hsize]{filename}}
\end{dispListing}
\end{docCommand}
\begin{docCommand}{licenseStatement}{}
A notice concerning the licence, e.g.\ ‘Released under an X licence.’
\end{docCommand}
\begin{docCommand}{licenseUrl}{}
The URL of the full licence text, if applicable.
\end{docCommand}
\end{frame}

\subsection{Composing your presentation}

The class is set up to ignore non-frame text in presentation mode, and ignore
frame titles in article mode. The idea is that you put additional commentary
outside frames, and it shows up in the transcript but not the slides. With
pictorial slides, you can wrap them in \env{figure} environments, and with
textual slides you can run them into the text of the commentary.

In article mode, article-only text is shown in a serif font,
while text that also appears on a slide is shown in a sans serif font,
so you can tell them apart.

\begin{frame}{Colours}
\begin{multicols}{3}[Both the class and standalone theme provide the RDA colour palette:]
\only<article>{\setlength{\parskip}{1ex}}%
\sample{rdayellow}\enspace\texttt{rdayellow}\par
\sample{rdagreen}\enspace \texttt{rdagreen}\par
\sample{rdabrown}\enspace \texttt{rdabrown}\par
\sample{rdagray}\enspace  \begin{tabular}{@{}l@{}}
  \texttt{rdagray}\\\texttt{rdagrey}
\end{tabular}\par
\sample{rdamidyellow}\enspace\texttt{rdamidyellow}\par
\sample{rdamidgreen}\enspace \texttt{rdamidgreen}\par
\sample{rdamidbrown}\enspace \texttt{rdamidbrown}\par
\sample{rdamidgray}\enspace  \begin{tabular}{@{}l@{}}
  \texttt{rdamidgray}\\\texttt{rdamidgrey}
\end{tabular}\par
\sample{rdalightyellow}\enspace\texttt{rdalightyellow}\par
\sample{rdalightgreen}\enspace \texttt{rdalightgreen}\par
\sample{rdalightbrown}\enspace \texttt{rdalightbrown}\par
\sample{rdalightgray}\enspace  \begin{tabular}{@{}l@{}}
  \texttt{rdalightgray}\\\texttt{rdalightgrey}
\end{tabular}\par
\end{multicols}
\end{frame}

\begin{frame}[fragile]{New frame options}
The class provides two new options for frames.

\begin{docKey}{background}{=\meta{filename}}{default \cs{defaultbgpicture}}
Use this option on its own to restore the usual slide background.
Use it with an image filename to use that image as the background instead.
\only<article>{\par
The class themes come with various backgrounds you can use:

\begin{itemize}
\item\bthm{RDA} (4:3 aspect ratio)
\begin{itemize}
\item \texttt{rda-bg-title1}
\item \texttt{rda-bg-title2}
\item \texttt{rda-bg-normal}\textsuperscript{\textdagger}
\end{itemize}
\item\bthm{RDA2020} (16:9 aspect ratio), backported to \bthm{RDA2016}
\begin{itemize}
\item \texttt{rda-bg-wmark}\textsuperscript{\textdagger}
\end{itemize}
\item\bthm{RDA2024} (16:9 aspect ratio)
\begin{itemize}
\item \texttt{rda24-title}
\item \texttt{rda24-part-L}, \texttt{rda24-part-R}
\item \texttt{rda24-chap-L}, \texttt{rda24-chap-R}
\item \texttt{rda24-sect-L}, \texttt{rda24-sect-R}
\item \texttt{rda24-body-L1}\textsuperscript{\textdagger},
  \texttt{rda24-body-L2}, \texttt{rda24-body-R1}, \texttt{rda24-body-R2}
\item \texttt{rda24-finale}
\end{itemize}
\end{itemize}

Renew \cs{defaultbgpicture} to choose a different default background.
The initial value of \cs{defaultbgpicture} is highlighted for each theme with
\textsuperscript{\textdagger}. If the \texttt{rda-bg-wmark} background is used,
an overlay is used to desaturate it.
}
\end{docKey}
\begin{docKey}{nobackground}{}{no value}
This gives the slide a plain white background. Initially, this option is
\emph{set} in the \bthm{RDA2016} theme and \emph{unset} in the other RDA themes.
\end{docKey}
\end{frame}

Note that the background image will be resized to fill the slide completely,
so to avoid distortion use an image that has approximately the right aspect ratio.
By default, \pkg{beamer} slides are 128\,mm $\times$ 96\,mm, which is a 4:3 ratio,
but a different ratio can be chosen with the \key{aspectratio} option to the
\pkg{beamer} class.

The above two options have no effect in article mode.

\begin{frame}[fragile]{Progress meter}
Slides show progress information at the bottom right in the form of a charge-style indicator.
To show progress as a fraction using frame numbers
\only<presentation>{(e.g. \insertframenumber\slash\inserttotalframenumber):}
\only<article>{(e.g. 10\slash 17):}
\begin{dispListing}
\setbeamertemplate{progress}[fraction]
\end{dispListing}

To show just the current frame number:
\begin{dispListing}
\setbeamertemplate{progress}[number]
\end{dispListing}

To show nothing at all:
\begin{dispListing}
\setbeamertemplate{progress}{}
\end{dispListing}
\end{frame}

To restore the default charge-style indicator:
\begin{dispListing}
\setbeamertemplate{progress}[charge]
\end{dispListing}


\subsection{Finishing off}

\begin{frame}[fragile]{Finishing off}
You can insert a closing slide in your presentation to parallel your opening slide.

\begin{docCommand}{finale}{\oarg{important note}}
The optional argument lets you add an important note, such as a key link or date, to the end of the slide. There is not much room in the original RDA theme, so be sparing if that's the one you use.

Wrap it in a bare frame, as you would for \cs{maketitle}:
\begin{tcolorbox}
\cs{begin}\brackets{frame}\\
\cs{finale}\\
\cs{end}\brackets{frame}
\end{tcolorbox}
\end{docCommand}
\end{frame}

In article mode, \cs{finale} inserts a postscript consisting of a horizontal line
followed by the licence and RDA group information or, failing that,
a brief statement about RDA.

\begin{frame}
\finale
\end{frame}

\mode<presentation>
\end{document}

\mode*
\StopEventually{^^A
  \PrintChanges
  \printindex
}

\newpage
\section{Implementation}

Sections marked `Class' appear in \texttt{rdaslides.cls}.
Sections marked `Theme' appear in one or more of \texttt{beamerthemeRDA.sty},
\texttt{beamerthemeRDA2016.sty}, \texttt{beamerthemeRDA2020.sty}, or
\texttt{beamerthemeRDA2024.sty}, as indicated in the text.

The final few sections each document a different separate helper package.

\setcounter{FancyVerbLine}{0}%
\DocInput{\jobname.dtx}
\end{document}
%</driver>
% \fi
% \iffalse
%<*class|theme|theme2|theme3|theme4>
% \fi
%
% \subsection{General dependencies}
%
% \subsubsection{Class and Theme}
%
% We will need the \pkg{etoolbox} package's patching utilities;
% \pkg{calc} is helpful for calculating lengths;
% \pkg{adjustbox} is helpful for precise positioning.
%
%    \begin{macrocode}
\RequirePackage{etoolbox, calc, adjustbox}
%    \end{macrocode}
%
% \iffalse
%</class|theme|theme2|theme3|theme4>
%<*theme|theme2|theme3|theme4>
% \fi
%
% \subsubsection{Theme only}
%
% We need \pkg{tikz} for drawing and layout.
%
%    \begin{macrocode}
\RequirePackage{tikz}

%    \end{macrocode}
%
% \iffalse
%</theme|theme2|theme3|theme4>
%<*class>
% \fi
%
% \subsection{Class: Options and mode switching}
%
% \begin{optionkey}{2013}
% \begin{optionkey}{2016}
% \begin{optionkey}{2020}
% \begin{optionkey}{2024}
% We provide keys for choosing between the available themes.
%
%    \begin{macrocode}

\def\rdaslides@theme{RDA}
\DeclareOption{2013}{\def\rdaslides@theme{RDA}}
\DeclareOption{2016}{\def\rdaslides@theme{RDA2016}}
\DeclareOption{2020}{\def\rdaslides@theme{RDA2020}}
\DeclareOption{2024}{\def\rdaslides@theme{RDA2024}}
%    \end{macrocode}
% \end{optionkey}
% \end{optionkey}
% \end{optionkey}
% \end{optionkey}
%
% \begin{optionkey}{set}
% Mode switching is delegated to \pkg{beamerswitch}, but we need to support the
% legacy \key{set} option.
%
%    \begin{macrocode}
\DeclareOption{set}{\PassOptionsToClass{article,alsobeamer}{beamerswitch}}
%    \end{macrocode}
% \end{optionkey}
%
% All other options are passed to \pkg{beamerswitch}.
%
%    \begin{macrocode}
\DeclareOption*{%
  \PassOptionsToClass{\CurrentOption}{beamerswitch}%
}
\ProcessOptions\relax

%    \end{macrocode}
%
% \subsection{Class: Setting options in loaded classes and packages}
%
% Now we load everything we need.
%
% It looks odd to have the \cs{inst} markers from \pkg{beamer} in the maths font, so
% we use the \key{textinst} option for \pkg{beamerswitch}.
%
%    \begin{macrocode}
\PassOptionsToClass{textinst}{beamerswitch}
%    \end{macrocode}
%
% In article mode, we default to 12pt text to keep it readable on the podium.
%
%    \begin{macrocode}
\PassOptionsToClass{a4paper,12pt}{article}
%    \end{macrocode}
%
% I anticipate using fonts with the necessary symbols for \pkg{beamer}, but which
% may clash with \pkg{amssymb}.
%
%    \begin{macrocode}
\PassOptionsToClass{noamssymb}{beamer}
\PassOptionsToPackage{noamssymb}{beamerarticle}
%    \end{macrocode}
%
% We provide the possibility of using colourful tables.
%
%    \begin{macrocode}
\PassOptionsToPackage{table}{xcolor}
\LoadClass{beamerswitch}

%    \end{macrocode}
%
% \subsection{Class: Layout, orthography and typography}
%
% In handout mode, we lay out two slides to an A4 page (by default).
%
%    \begin{macrocode}
\handoutlayout{paper=a4paper,nup=2,pnos}
%    \end{macrocode}
%
% I find slide titles somewhat intrusive in article mode (unless repurposed as
% float captions, say), so we turn them off. I also make use of the more
% \pkg{beamer}-esque \cs{maketitle} routine provided by \pkg{beamerswitch}.
%
%    \begin{macrocode}
\articlelayout{maketitle,frametitles=none}
%    \end{macrocode}
%
% Left and right margins of 3cm and top and bottom margins of 2.5cm
% give a typeblock with an approximate golden aspect ratio.
%
%    \begin{macrocode}
\mode<article>{
  \RequirePackage[hmargin=3cm,vmargin=2.5cm]{geometry}
}
%    \end{macrocode}
%
% I anticipate using fonts with professional features.
%
%    \begin{macrocode}
\mode<presentation>{
  \usefonttheme{professionalfonts}
}
%    \end{macrocode}
%
% URLs should not use a radically different font.
%
%    \begin{macrocode}
\urlstyle{same}
%    \end{macrocode}
%
% I use British orthography. If there is a demand for it, I might convert the
% following lines to respect class options instead of a hard-coded language, but
% it is easy enough to patch them with \cs{PassOptionsToPackage} before loading
% the class.
%
%    \begin{macrocode}
\RequirePackage[british]{babel}
\RequirePackage[british,cleanlook]{isodate}
%    \end{macrocode}
%
% We take advantage of \pkg{microtype} enhancements. We defer loading it until the
% end of the preamble in case the author loads \pkg{fontspec}.
%
%    \begin{macrocode}
\AtEndPreamble{%
  \RequirePackage{microtype}%
}
%    \end{macrocode}
%
% In article mode, to give the speaker an impression of what appears on the
% slides and what doesn't, we switch to sans serif for slide contents.
%
%    \begin{macrocode}
\addtobeamertemplate{frame begin}{}{\sffamily}
\addtobeamertemplate{frame end}{\rmfamily}{}
%    \end{macrocode}
%
% When reading from a podium, some extra whitespace helps me keep track of where
% I am. So we switch to using Web-style paragraphs.
%
%    \begin{macrocode}
\mode<article>{%
  \setlength{\parindent}{0pt}%
  \setlength{\parskip}{1em plus 0.2em minus 0.2em}%
%    \end{macrocode}
%
% But this makes lists a bit too open, so we tighten them up again by reducing
% the \cs{topsep} to zero. We do this by injecting the appropriate code into the
% second argument of \cs{list} where this appears in the definition of the three
% main list environments.
%
%    \begin{macrocode}
  \patchcmd{\itemize}{\def}{\topsep\z@\def}%
    {\wlog{Patching itemize succeeded}}%
    {\wlog{Patching itemize failed}}
  \patchcmd{\@enum@}{\def}{\topsep\z@\def}%
    {\wlog{Patching enumerate succeeded}}%
    {\wlog{Patching enumerate failed}}
  \patchcmd{\description}{\labelwidth\z@}{\labelwidth\z@\topsep\z@}%
    {\wlog{Patching description succeeded}}%
    {\wlog{Patching description failed}}
%    \end{macrocode}
%
% We make captions easier to spot. We try to make the placement of the final
% period more intelligent using \cs{@addpunct} from \pkg{amsthm}.
%
%    \begin{macrocode}
  \RequirePackage[format=hang,justification=raggedright,labelfont=bf]{caption}
  \RequirePackage{amsthm}
  \DeclareCaptionTextFormat{condperiod}{#1\@addpunct{.}}
  \captionsetup{textformat=condperiod}
%    \end{macrocode}
%
% We also make footnotes look a little neater.
%
%    \begin{macrocode}
  \RequirePackage[hang,multiple,bottom]{footmisc}
  \setlength{\footnotemargin}{1em}
}

%    \end{macrocode}
%
% \iffalse
%</class>
%<*theme|theme2|theme3|theme4>
% \fi
%
% \subsection{Theme: Presentation font theme}
%
% Here are font sizes and weights common to all themes.
%
%    \begin{macrocode}
\setbeamerfont{alerted text}{series=\bfseries}
\setbeamerfont{frametitle}{size=\LARGE}
\setbeamerfont{framesubtitle}{size=\large}
%    \end{macrocode}
%
% \iffalse
%</theme|theme2|theme3|theme4>
%<*theme>
% \fi
%
% There are small differences in the fonts used for RDA group information
% (\key{headline} and \key{subheadline}), and text in the \key{footline} (e.g.\@
% hashtag, progress indicator). Here are the settings for the \bthm{RDA} theme:
%
%    \begin{macrocode}
\setbeamerfont{headline}{size=\LARGE}
\setbeamerfont{subheadline}{size=\small,shape=\scshape}
\setbeamerfont{footline}{size=\scriptsize}
%    \end{macrocode}
%
% \iffalse
%</theme>
%<*theme2>
% \fi
%
% Here are the settings for the \bthm{RDA2016} theme:
%
%    \begin{macrocode}
\setbeamerfont{headline}{size=\LARGE}
\setbeamerfont{subheadline}{size=\small,shape=\scshape}
\setbeamerfont{footline}{size=\tiny}
%    \end{macrocode}
%
% \iffalse
%</theme2>
%<*theme3>
% \fi
%
% The \bthm{RDA2020} theme also has larger title text (the title slide is roomier):
%
%    \begin{macrocode}
\setbeamerfont{headline}{size=\LARGE}
\setbeamerfont{subheadline}{size=\large}
\setbeamerfont{footline}{size=\scriptsize}
\setbeamerfont{title}{size=\LARGE}
%    \end{macrocode}
%
% \iffalse
%</theme3>
%<*theme4>
% \fi
%
% The \bthm{RDA2024} theme needs a bold title; unlike the others, it has a unique
% style for the thank you message on the finale slide.
%
%    \begin{macrocode}
\setbeamerfont{headline}{parent=date}
\setbeamerfont{subheadline}{size=\small,shape=\scshape}
\setbeamerfont{footline}{size=\tiny}
\setbeamerfont{title}{series=\bfseries}
\setbeamerfont{negdate}{series=\bfseries}
\setbeamerfont{thanks}{parent=title,size=\huge}
%    \end{macrocode}
%
% \iffalse
%</theme4>
%<*class|theme|theme2|theme3|theme4>
% \fi
%
% \subsection{Theme: Presentation colour theme}
%
% \subsubsection{Common code}
%
% We load the RDA colour palette as defined in section~\ref{sec:rdacolors}.
%
%    \begin{macrocode}

\RequirePackage{rdacolors}
%    \end{macrocode}
%
% \iffalse
%</class|theme|theme2|theme3|theme4>
%<*theme|theme2|theme3|theme4>
% \fi
%
% Here are the basic colours:
%
%    \begin{macrocode}
\setbeamercolor{normal text}{bg=white,fg=black}
\setbeamercolor{alerted text}{fg=rdagreen}
\setbeamercolor{example text}{fg=rdagrey}
\setbeamercolor{structure}{fg=rdabrown}
%    \end{macrocode}
%
% \iffalse
%</theme|theme2|theme3|theme4>
%<*theme>
% \fi
%
% \subsubsection{\bthm{RDA} theme}
%
% Here are the colours for the inner theme elements:
%
%    \begin{macrocode}
\setbeamercolor{title}{fg=white}
\setbeamercolor{subtitle}{fg=white}
\setbeamercolor{author}{fg=white}
\setbeamercolor{institute}{fg=white}
\setbeamercolor{date}{fg=white}
\setbeamercolor{item}{fg=rdagreen}
\setbeamercolor{subitem}{fg=rdabrown}
\setbeamercolor{subsubitem}{fg=rdayellow}
%    \end{macrocode}
%
% Here are the colours for the outer theme elements:
%
%    \begin{macrocode}
\setbeamercolor{frametitle}{bg=,fg=rdagreen}
\setbeamercolor{headline}{parent=frametitle}
\setbeamercolor{footline}{bg=,fg=rdabrown}
\setbeamercolor{progress}{bg=rdalightyellow,fg=rdamidbrown}

%    \end{macrocode}
%
% \iffalse
%</theme>
%<*theme2>
% \fi
%
% \subsubsection{\bthm{RDA2016} theme}
%
% Here are the colours for the inner theme elements:
%
%    \begin{macrocode}
\setbeamercolor{title}{parent=normal text}
\setbeamercolor{subtitle}{parent=normal text}
\setbeamercolor{author}{parent=normal text}
\setbeamercolor{institute}{parent=normal text}
\setbeamercolor{date}{parent=normal text}
\setbeamercolor{item}{fg=rdagreen}
\setbeamercolor{subitem}{fg=rdabrown}
\setbeamercolor{subsubitem}{fg=rdayellow}
%    \end{macrocode}
%
% Here are the colours for the outer theme elements:
%
%    \begin{macrocode}
\setbeamercolor{frametitle}{bg=,fg=rdagreen}
\setbeamercolor{headline}{parent=frametitle}
\setbeamercolor{footline}{bg=rdagreen,fg=white}
\setbeamercolor{progress}{bg=white,fg=rdamidgreen}
%    \end{macrocode}
% \iffalse
%</theme2>
%<*theme3>
% \fi
%
% \subsubsection{\bthm{RDA2020} theme}
%
% Here are the colours for the inner theme elements:
%
%    \begin{macrocode}
\setbeamercolor{title}{parent=frametitle}
\setbeamercolor{subtitle}{parent=frametitle}
\setbeamercolor{author}{parent=normaltext}
\setbeamercolor{institute}{parent=normaltext}
\setbeamercolor{date}{parent=normaltext}
\setbeamercolor{item}{fg=rdabrown}
\setbeamercolor{subitem}{fg=rdagreen}
\setbeamercolor{subsubitem}{fg=rdabrown}
%    \end{macrocode}
%
% Here are the colours for the outer theme elements:
%
%    \begin{macrocode}
\setbeamercolor{frametitle}{bg=,fg=rdagreen}
\setbeamercolor{headline}{parent=frametitle}
\setbeamercolor{footline}{bg=,fg=white}
\setbeamercolor{progress}{bg=white,fg=rdalightbrown}
%    \end{macrocode}
% \iffalse
%</theme3>
%<*theme4>
% \fi
%
% \subsubsection{\bthm{RDA2024} theme}
%
% Here are the colours for the inner theme elements:
%
%    \begin{macrocode}
\setbeamercolor{title}{fg=black}
\setbeamercolor{subtitle}{parent=title,fg=rdagray}
\setbeamercolor{author}{parent=title}
\setbeamercolor{institute}{parent=title}
\setbeamercolor{date}{parent=title}
\setbeamercolor{negdate}{fg=white}
\setbeamercolor{item}{fg=rdabrown}
\setbeamercolor{subitem}{fg=rdagreen}
\setbeamercolor{subsubitem}{fg=rdayellow}
\setbeamercolor{thanks}{fg=rdagreen}
%    \end{macrocode}
%
% Here are the colours for the outer theme elements:
%
%    \begin{macrocode}
\setbeamercolor{frametitle}{bg=,fg=black}
\setbeamercolor{framesubtitle}{bg=,fg=rdagray}
\setbeamercolor{headline}{bg=,fg=rdagray}
\setbeamercolor{footline}{bg=,fg=rdagreen}
\setbeamercolor{progress}{bg=white,fg=rdamidgreen}
%    \end{macrocode}
%
% \iffalse
%</theme4>
%<*class>
% \fi
%
% \subsection{Class: Metadata handling}
%
% \begin{macro}{occasion}
% \begin{macro}{insertoccasion}
% We define some new elements of metadata.
% The \cs{occasion} command is used to record the event at which the presentation
% is given. The content is available if provided via \cs{insertoccasion}.
%
%    \begin{macrocode}
\newcommand*{\occasion}[1]{%
  \def\insertoccasion{#1}%
  \subject{Presentation given at #1}%
  \mode<article>{%
    \AtBeginDocument{\hypersetup{pdfsubject={Presentation given at #1}}}%
  }%
}
%    \end{macrocode}
% \end{macro}
% \end{macro}
%
% \begin{macro}{hashtag}
% \begin{macro}{inserthashtag}
% The \cs{hashtag} command is used to suggest a hashtag people can use to share
% this presentation on social media. The content is available if provided via
% \cs{inserthashtag}
%
%    \begin{macrocode}
\newcommand*{\hashtag}[1]{%
  \newcommand*{\inserthashtag}{\href{https://bsky.app/hashtag/#1}{\##1}}%
}
\newcommand{\RDAFormatHandle}[1]{ \textbar\ \href{https://twitter.com/#1}{@#1}}
\newcommand{\handles}[1]{\def\handlecsv{{#1}}}
\handles{rda\_europe, RDA\_US}
%    \end{macrocode}
% \end{macro}
% \end{macro}
%
% \begin{macro}{authorurl}
% \begin{macro}{insertauthorurl}
% The \cs{authorurl} command is used to give a URL where more information about
% the speaker is available. The content is available if provided via
% \cs{insertauthorurl}
%
%    \begin{macrocode}
\newcommand*{\authorurl}[1]{%
  \def\insertauthorurl{#1}%
}
%    \end{macrocode}
% \end{macro}
% \end{macro}
%
% In case the author forgets to include some key metadata, we provide some
% safety values to allow the document to compile.
%
%    \begin{macrocode}
\def\@title{Please provide a title}
\def\@author{Please specify the author}
\edef\@date{\the\year-\ifnum\month<10 0\fi\the\month-\ifnum\day<10 0\fi\the\day}
%    \end{macrocode}
%
% We add these new elements to the \cs{maketitle} routine in article mode.
%
%    \begin{macrocode}
\mode<article>{
  \patchcmd{\@maketitle}{%
    \@author
  }{%
    \ifundef{\insertauthorurl}{\@author}{\href{\insertauthorurl}{\@author}}%
  }{}{}
  \patchcmd{\@maketitle}{%
    \large \@date
  }{%
    \large
    \ifdefvoid{\insertoccasion}{}{\insertoccasion, }%
    \printdate{\@date}%
  }{}{}
}

%    \end{macrocode}
%
% \iffalse
%</class>
%<*theme|theme2|theme3|theme4>
% \fi
%
% \subsection{Theme: Presentation outer theme}
%
% The outer theme means things like the background, head, foot, and frame title.
%
% \subsubsection{Utility settings}
%
% We introduce a toggle, |titlepage|, that can be used to provide a
% different layout depending on whether this is a normal or a title slide.
% Another toggle, |finalepage|, is used to distinguish whether the regular
% title background or the finale background is used (in some themes this is the
% same).
%
%    \begin{macrocode}

\newtoggle{titlepage}
\newtoggle{finalepage}
%    \end{macrocode}
%
% We provide another toggle, |bgpicture|, which is used to decide whether
% to use a blank background (‘false’) or an image (‘true’).
%
%    \begin{macrocode}
\newtoggle{bgpicture}
%    \end{macrocode}
%
% \iffalse
%</theme|theme2|theme3|theme4>
%<*theme2>
% \fi
%
% The \bthm{RDA2016} theme has a blank background.
%
%    \begin{macrocode}
\togglefalse{bgpicture}
%    \end{macrocode}
%
% \iffalse
%</theme2>
%<*theme|theme3|theme4>
% \fi
%
% The other themes, however, use some distinctive slide backgrounds.
%
%    \begin{macrocode}
\toggletrue{bgpicture}
%    \end{macrocode}
%
% \iffalse
%</theme|theme3|theme4>
%<*theme>
% \fi
%
% \begin{macro}{defaultbgpicture}
% The \bthm{RDA} theme defaults to this background.
%
%    \begin{macrocode}
\newcommand{\defaultbgpicture}{rda-bg-normal}
%    \end{macrocode}
%
% \iffalse
%</theme>
%<*theme2|theme3>
% \fi
%
% The \bthm{RDA2020} theme defaults to this background, which requires some
% special handling (see below). It is also used as the default if the background
% is switched on in the \bthm{RDA2016} theme.
%
%    \begin{macrocode}
\newcommand{\defaultbgpicture}{rda-bg-wmark}
%    \end{macrocode}
%
% \iffalse
%</theme2|theme3>
%<*theme4>
% \fi
%
% The \bthm{RDA2024} theme defaults to this background. There are at least
% another three good candidates, but this one seems to be preferred in the
% RDA Style Guide 2024.
%
%    \begin{macrocode}
\newcommand{\defaultbgpicture}{rda24-body-L1}
%    \end{macrocode}
% \end{macro}
%
% \iffalse
%</theme4>
%<*theme|theme2|theme3|theme4>
% \fi
%
% \begin{macro}{bgpicture}
% \begin{optionkey}{background}
% \begin{optionkey}{nobackground}
% We allow the user to select an image to use as a slide background, defaulting
% to the theme-dependent \cs{defaultbgpicture} as defined above.
% The method used is to save the filename of the image to \cs{bgpicture}, and
% let the user change it with a frame option, \key{background}. The image can be
% turned off entirely by issuing with the frame option \key{nobackground}.
%
%    \begin{macrocode}
\let\bgpicture\defaultbgpicture
\define@key{beamerframe}{background}[\defaultbgpicture]{%
  \xdef\bgpicture{#1}\toggletrue{bgpicture}%
}
\define@key{beamerframe}{nobackground}[true]{%
  \togglefalse{bgpicture}%
}
%    \end{macrocode}
% \end{optionkey}
% \end{optionkey}
% \end{macro}
%
% \subsubsection{Background imagery template}
%
% We stretch the aspect ratio of the background to fill the slide. The
% \texttt{rda-bg-wmark} image is handled specially, as it is a full-saturation
% tiling pattern: it is first scaled to the page height, then (if necessary)
% scaled up further to fill the width and cropped, then desaturated with a white
% overlay.
%
%    \begin{macrocode}
\setbeamertemplate{background canvas}{%
  \iftoggle{bgpicture}{%
    \ifdefstring{\bgpicture}{rda-bg-wmark}{%
      \adjustimage{%
        height=\paperheight,
        min width=\paperwidth,
        center=\paperwidth}{\bgpicture}%
      \begin{adjustbox}{llap}%
        \begin{tikzpicture}
          \fill[white,ultra nearly opaque]
            (0mm,0mm) rectangle (\paperwidth,\paperheight);
        \end{tikzpicture}%
      \end{adjustbox}%
    }{%
      \adjustimage{height=\paperheight,width=\paperwidth}{\bgpicture}%
    }%
%    \end{macrocode}
%
% \iffalse
%</theme|theme2|theme3|theme4>
%<*theme|theme2|theme4>
% \fi
%
% In the \bthm{RDA}, \bthm{RDA2016}, and \bthm{RDA2024} themes, that's all we
% need to do on the background canvas.
%
%    \begin{macrocode}
  }{}%
}
%    \end{macrocode}
%
% \iffalse
%</theme|theme2|theme4>
%<*theme3>
% \fi
%
% In the \bthm{RDA2020} theme, we also add the bottom brown strip to the 
% background canvas and add the logo to the background.
%
%    \begin{macrocode}
    \begin{adjustbox}{llap}%
      \begin{tikzpicture}
        \useasboundingbox
          (0,0) rectangle (\paperwidth,\paperheight);
        \fill[rdabrown] (0,0) rectangle (\paperwidth,6.567mm);
      \end{tikzpicture}%
    \end{adjustbox}%
  }{%
    \begin{tikzpicture}
      \useasboundingbox
        (0,0) rectangle (\paperwidth,\paperheight);
      \fill[rdabrown] (0,0) rectangle (\paperwidth,6.567mm);
    \end{tikzpicture}%
  }%
}
\setbeamertemplate{background}{%
  \begin{tikzpicture}[overlay,remember picture]
    \node[anchor=north west,xshift=1.7mm,yshift=-4.11mm,inner sep=0mm]
    at (current page.north west)
    {\adjustimage{width=25.5mm}{rda-logo-notext}};
  \end{tikzpicture}%
}
%    \end{macrocode}
%
% \iffalse
%</theme3>
%<*theme|theme3|theme4>
% \fi
%
% To prevent the settings for these switches persisting between slides, we reset
% them as part of the set-up of \env{frame} environments. Here's the reset for
% the \bthm{RDA}, \bthm{RDA2020}, and \bthm{RDA2024} themes:
%
%    \begin{macrocode}
\preto\beamer@reseteecodes{\setkeys{beamerframe}{background}\togglefalse{titlepage}}
%    \end{macrocode}
%
% \iffalse
%</theme|theme3|theme4>
%<*theme2>
% \fi
%
% Here's the reset for the \bthm{RDA2016} theme:
%
%    \begin{macrocode}
\preto\beamer@reseteecodes{\setkeys{beamerframe}{nobackground}\togglefalse{titlepage}}
%    \end{macrocode}
%
% \subsubsection{Headline (RDA group information) template}
% \label{sec:headline}
%
% Normal slides do not have a headline, but we can use the \texttt{headline} template to
% add (optionally) details of the RDA group to the title slides. In the
% \bthm{RDA2016} theme, we also need to use it to add the big RDA logo since we
% don't have the background image that contains it.
%
% If provided, the group logo goes in a 16mm box in the top left;
% the main logo is central, and at this scale has roughly a 3mm white border;
% and if provided, the group name and type go in a 42mm box in the top right.
% All are top-aligned.
%
%    \begin{macrocode}
\defbeamertemplate*{headline}{RDA headline}{%
  \iftoggle{titlepage}{
    \begin{beamercolorbox}[sep=0pt]{headline}
      \ifdefvoid{\rdaGroupLogo}{%
        \adjustbox{set vsize={0pt}{0pt}}{}%
      }{%
        \adjustbox{width=16mm,margin=3mm,raise=-\height,left=0pt}{%
          \rdaGroupLogo}%
      }\hfill
      \adjustimage{height=33.8mm,raise=-\height}{rda-logo}\hfill
      \ifdefvoid{\rdaGroupLogo}{%
        \adjustbox{set vsize={0pt}{0pt}}{}%
      }{%
        \adjustbox{minipage=42mm,margin=3mm,raise=-\height,right=0pt}{%
          \raggedleft
          {\linespread{0.75}\selectfont\rdaGroupName\par}%
          \ifdefvoid{\rdaGroupType}{}{%
            \smallskip{\usebeamerfont{subheadline}\rdaGroupType\par}}}}%
    \end{beamercolorbox}%
  }{}%
}
%    \end{macrocode}
%
% \iffalse
%</theme2>
%<*theme3>
% \fi
%
% In the \bthm{RDA2020} theme, the group logo (if provided) again goes in a
% 16mm box, but this time it sits top right, aligned vertically centred
% relative to the RDA logo that was added to the background template above.
% The group name and type, looking very similar to a frame title and subtitle,
% are left-aligned in a box that is right-aligned against the group logo; its
% maximum width is determined by the 1.7mm margins, the 25.5mm and 16mm logos,
% and two 1em spaces adjacent to the logos.
%
%    \begin{macrocode}
\defbeamertemplate*{headline}{RDA headline}{%
  \iftoggle{titlepage}{
    \begin{beamercolorbox}[sep=1.7mm]{headline}
      \rule{0pt}{8.485mm}\hfill
      \ifdefvoid{\rdaGroupName}{}{%
        \begin{adjustbox}%
        { varwidth=\dimexpr\paperwidth-44.9mm-2em\relax
        , valign=M
        , set vsize={0pt}{0pt}
        }%
          {\linespread{0.75}\selectfont\rdaGroupName\par}%
          \ifdefvoid{\rdaGroupType}{}{%
            {\usebeamerfont{subheadline}\rdaGroupType\par}}%
        \end{adjustbox}%
      }\quad
      \ifdefvoid{\rdaGroupLogo}{}{%
        \adjustbox{width=16mm,valign=M,set vsize={0pt}{0pt}}{\rdaGroupLogo}%
      }%
    \end{beamercolorbox}%
  }{}%
}
%    \end{macrocode}
%
% \iffalse
%</theme3>
%<*theme4>
% \fi
%
% In the \bthm{RDA2024} theme, the group information block is at the top right,
% but the design here has to cope with the RDA logo being absent on the title
% slide and present on the finale slide, so the group logo has to shunt over.
% The top left is already busy on the finale slide so we avoid it.
%
%    \begin{macrocode}
\defbeamertemplate*{headline}{RDA headline}{%
  \ifboolexpr{
    togl {titlepage}
    and not test {\ifdefvoid{\rdaGroupName}}
  }{%
    \begin{beamercolorbox}[sep=0pt,right]{headline}%
      \begin{adjustbox}%
      { minipage={0.292\paperwidth}
      , margin={0.03\paperwidth} {0.038\paperheight}
      }\raggedleft
        \adjustbox{set vsize={0.065\paperheight}{0.03\paperheight}}{%
          \ifdefvoid{\rdaGroupLogo}{}{%
            \adjustbox{width=0.065\paperheight,valign=B}{\rdaGroupLogo}}%
        }%
        \iftoggle{finalepage}{\quad\hbox to 0.074\paperwidth{}}{}%
        \par
        {\linespread{0.75}\selectfont\rdaGroupName\par}
        \ifdefvoid{\rdaGroupType}{}{%
          {\usebeamerfont{subheadline}\rdaGroupType\par}}%
      \end{adjustbox}%
    \end{beamercolorbox}%
  }{}%
}
%    \end{macrocode}
%
% \iffalse
%</theme4>
%<*theme>
% \fi
%
% The \texttt{headline} template in the \bthm{RDA} theme is almost the same as the one
% for the \bthm{RDA2016} theme, except that the central logo doesn't need to be
% added (it's in the background image), and the group logo and name are
% vertically centred in the top 24mm of the slide instead of being top-aligned.
%
%    \begin{macrocode}
\defbeamertemplate*{headline}{RDA headline}{%
  \iftoggle{titlepage}{
    \begin{beamercolorbox}[sep=0pt,center]{headline}
      \begin{adjustbox}{minipage=[b][24mm][c]{\paperwidth-6mm}}%
        \ifdefvoid{\rdaGroupLogo}{%
          \adjustbox{set vsize={0pt}{0pt}}{}%
        }{%
          \adjustbox{width=16mm,valign=M,left=0pt}{%
            \rdaGroupLogo}%
        }\hfill
        \ifdefvoid{\rdaGroupLogo}{%
          \adjustbox{set vsize={0pt}{0pt}}{}%
        }{%
          \adjustbox{minipage=42mm,valign=M,right=0pt}{%
            \raggedleft
            {\linespread{0.75}\selectfont\rdaGroupName\par}%
            \ifdefvoid{\rdaGroupType}{}{%
              \smallskip{\usebeamerfont{subheadline}\rdaGroupType\par}}}}%
      \end{adjustbox}%
    \end{beamercolorbox}%
  }{}%
}
%    \end{macrocode}
%
% \subsubsection{Frame title template}
%
% None of the title page designs are amenable to the inclusion of a frame title,
% so we routinely ignore frame titles while the \texttt{titlepage} toggle is
% true.
%
% In the \bthm{RDA} theme, the frame title block in the default background image
% curves down in the middle, so we need to tell \pkg{beamer} to make its
% \texttt{frametitle} template a little deeper to cover it. The following code
% mimics the default, but adds a \key{dp} option to the \env{beamercolorbox}
% environment.
%
%    \begin{macrocode}
\defbeamertemplate*{frametitle}{RDA theme}{%
  \ifbeamercolorempty[bg]{frametitle}{}{\nointerlineskip}%
  \iftoggle{titlepage}{}{%
    \@tempdima=\textwidth%
    \advance\@tempdima by\beamer@leftmargin%
    \advance\@tempdima by\beamer@rightmargin%
    \begin{beamercolorbox}[sep=0.3cm,wd=\the\@tempdima,dp=6mm]{frametitle}
      \usebeamerfont{frametitle}%
      \vbox{}\vskip-1ex%
      \strut\insertframetitle\strut\par%
      {%
        \ifdefvoid{\insertframesubtitle}{}{%
          \usebeamerfont{framesubtitle}%
          \usebeamercolor[fg]{framesubtitle}%
          \insertframesubtitle\strut\par
        }%
      }%
      \vskip-1ex%
      \if@tempswa\else\vskip-.3cm\fi%
    \end{beamercolorbox}%
  }%
}
%    \end{macrocode}
%
% \iffalse
%</theme>
%<*theme2>
% \fi
%
% Here is the corresponding version for the \bthm{RDA2016} theme.
% The title block begins with the RDA logo flush to the slide edge.
%
%    \begin{macrocode}
\defbeamertemplate*{frametitle}{RDA theme}{%
  \ifbeamercolorempty[bg]{frametitle}{}{\nointerlineskip}%
  \iftoggle{titlepage}{}{%
    \begin{beamercolorbox}[sep=0pt,wd=\paperwidth]{frametitle}
      \usebeamerfont{frametitle}%
      \hangindent=17.5mm
      \hangafter=1
      \adjustimage{width=14.5mm,raise={-0.2\height}{7.3mm}{0pt}}{rda-logo}%
      \hspace{3mm}%
      \insertframetitle\par%
      {%
        \ifdefvoid{\insertframesubtitle}{}{%
          \usebeamerfont{framesubtitle}%
          \usebeamercolor[fg]{framesubtitle}%
          \insertframesubtitle\strut\par
        }%
      }%
    \end{beamercolorbox}%
  }%
}
%    \end{macrocode}
%
% \iffalse
%</theme2>
%<*theme3>
% \fi
%
% In the \bthm{RDA2020} theme, the template is similar in intent to the one in
% \bthm{RDA2016} theme, but the logo is in the background, bigger and lower,
% so the frame title block is vertically centred next to where it should be.
%
%    \begin{macrocode}
\defbeamertemplate*{frametitle}{RDA theme}{%
  \ifbeamercolorempty[bg]{frametitle}{}{\nointerlineskip}%
  \iftoggle{titlepage}{}{%
    \begin{beamercolorbox}[sep=0pt,wd=\paperwidth]{frametitle}
      \usebeamerfont{frametitle}%
      \hangindent=30mm
      \hangafter=1
      \rule{0pt}{9.7mm}%
      \hspace{30mm}%
      \begin{adjustbox}{varwidth=\dimexpr\paperwidth-30mm-1em\relax,valign=M}
      \strut\insertframetitle\par%
      {%
        \ifdefvoid{\insertframesubtitle}{}{%
          \usebeamerfont{framesubtitle}%
          \usebeamercolor[fg]{framesubtitle}%
          \insertframesubtitle\strut\par
        }%
      }%
      \end{adjustbox}
    \end{beamercolorbox}%
  }%
}
%    \end{macrocode}
%
% \iffalse
%</theme3>
%<*theme4>
% \fi
%
% In the \bthm{RDA2024} theme, the logo is similar in size to the one in the
% \bthm{RDA2016} theme, but further inset, and with the option of being on
% either the left or right. For consistency across designs, we opt for centre
% alignment.
%
%    \begin{macrocode}
\defbeamertemplate*{frametitle}{RDA theme}{%
  \ifbeamercolorempty[bg]{frametitle}{}{\nointerlineskip}%
  \iftoggle{titlepage}{}{%
    \begin{adjustbox}{minipage=[c][0.14\paperheight]{0.73\paperwidth},center}%
      \begin{beamercolorbox}[sep=0pt,center]{frametitle}
        \linespread{0.75}\selectfont\insertframetitle\par%
        {%
          \ifdefvoid{\insertframesubtitle}{}{%
            \usebeamerfont{framesubtitle}%
            \usebeamercolor[fg]{framesubtitle}%
            \insertframesubtitle\strut\par
          }%
        }%
      \end{beamercolorbox}%
    \end{adjustbox}%
  }%
}
%    \end{macrocode}
%
% \iffalse
%</theme4>
%<*theme|theme2|theme3|theme4>
% \fi
%
% By default \pkg{beamer} uses Roman numerals for split frames.
% Across all themes, we change this to Arabic numerals.
%
%    \begin{macrocode}
\setbeamertemplate{frametitle continuation}{%
  (\insertcontinuationcount)%
}
%    \end{macrocode}
%
% \subsubsection{Progress indicators}
%
% We provide a variety of progress indicators.
%
% \begin{macro}{ProgressBar}
% The default is a charge-style indicator.
% The mathematics here makes sure the indicator start at 0\% on the
% first slide and ends up at 100\% on the last, with linear variation
% between. The case of a single slide is handled differently to prevent
% division by zero.
%
%    \begin{macrocode}
\newcommand{\ProgressBar}{%
  \ifbeamer@inappendix\relax\else
    \begin{tikzpicture}[thin,line join=round]
    \pgfmathparse{equal(\insertmainframenumber,1)}%
    \ifnum1=\pgfmathresult
    \def\PerCentComplete{12}%
    \else
    \pgfmathqparse{12pt * ((\insertframenumber pt - 1pt) /
      (\insertmainframenumber pt - 1pt))}%
    \let\PerCentComplete\pgfmathresult
    \fi
    \draw[fg,fill=bg] (0mm,0mm) rectangle (12mm,1.35mm);
    \filldraw[fg] (0mm,0mm) rectangle (\PerCentComplete mm,1.35mm);
    \end{tikzpicture}%
  \fi
}
\defbeamertemplate*{progress}{charge}{\ProgressBar}
%    \end{macrocode}
% \end{macro}
%
% As an alternative, progress can be shown as a fraction of frame number over
% total number of frames. We make sure it takes up the same space on all slides
% as it does on the last main slide.
%
%    \begin{macrocode}
\newlength{\progresswidth}
\ifdef{\g@addto@macro}{\g@addto@macro\beamer@lastminutepatches{%
  \settowidth{\progresswidth}{%
  \usebeamerfont{footline}\insertmainframenumber}}}{}%
\defbeamertemplate{progress}{fraction}{%
  \usebeamercolor{footline}\color{fg}%
  \ifbeamer@inappendix
    \adjustbox{right=\progresswidth}{\insertframenumber}%
    \phantom{\slash}\hspace*{\progresswidth}%
  \else
    \adjustbox{right=\progresswidth}{\insertframenumber}\slash
    \adjustbox{left=\progresswidth}{\insertmainframenumber}%
  \fi
}
%    \end{macrocode}
%
% Another possibility is just displaying the frame number.
%
%    \begin{macrocode}
\defbeamertemplate{progress}{number}{%
  \usebeamercolor{footline}\color{fg}%
  \adjustbox{right=\progresswidth}{\insertframenumber}%
}
%    \end{macrocode}
%
% \subsubsection{Footline template}
%
% The \pkg{beamer} \texttt{footline} template lays out elements along the bottom
% of the screen.
%
% \iffalse
%</theme|theme2|theme3|theme4>
%<*theme>
% \fi
%
% In the \bthm{RDA} theme, we set the height of the footline to 8mm (6mm with
% 1mm padding top and bottom) so it fits over the RDA logo in the standard
% background image.
%
% We use a typical layout: a left-aligned box at the left, a centre-aligned box
% in the middle and a right-aligned box at the right. The left hand box is
% occupied by the RDA strapline in the standard background image, so we only use
% it on the title slide, for the \pkg{beamer} title graphic.
%
%    \begin{macrocode}
\setbeamertemplate{footline}{%
  \hypersetup{urlcolor=fg}%
  \adjustbox{minipage=[b][6mm][t]{\dimexpr0.3\paperwidth-3mm\relax},margin=3mm 1mm,left=0pt}{%
    \iftoggle{titlepage}{%
      \usebeamercolor[fg]{titlegraphic}\inserttitlegraphic
    }{}%
  }%
  \hfill
%    \end{macrocode}
%
% We use the centre box (on all slides) for the hashtag, if given. The title
% slide has a darker background so requires a lighter text colour.
%
%    \begin{macrocode}
  \adjustbox{minipage=[b][6mm][t]{0.4\paperwidth},margin=0pt 1mm}{\centering
    \usebeamerfont{footline}%
    \iftoggle{titlepage}{\usebeamercolor[fg]{date}}{\usebeamercolor[fg]{footline}}%
    \ifdefvoid{\inserthashtag}{}{\inserthashtag}\strut\par
  }%
  \hfill
%    \end{macrocode}
%
% On the title slide, we use the right hand box for the licence image.
% In the standard background image the right hand box is occupied by the RDA
% logo, but since it is shorter than the strapline on the left, we can balance
% it with a progress indicator.
%
%    \begin{macrocode}
  \adjustbox{minipage=[b][6mm][t]{\dimexpr0.3\paperwidth-3mm\relax},margin=3mm 1mm,right=0pt}{%
    \iftoggle{titlepage}{%
      %\adjustbox{minipage=18mm,margin=3mm 0pt,raise=2mm}{%
      \adjustbox{width=18mm,right}{%
        \ifdefvoid{\licenseLogo}{}{%
          \ifdefvoid{\licenseUrl}{\licenseLogo}{\href{\licenseUrl}{\licenseLogo}}%
        }%
      }%
    }{%
      \strut\usebeamertemplate*{progress}
    }%
  }%
}
%    \end{macrocode}
%
% \iffalse
%</theme>
%<*theme2>
% \fi
%
% In the \bthm{RDA2016} theme, the footline is the same on all slides, and
% consists of a green strip containing left-, centre-, and right-aligned boxes.
%
%    \begin{macrocode}
\setbeamertemplate{footline}{%
  \makebox[0pt][l]{%
    \textcolor{rdamidgreen}{\rule[3.25mm]{\paperwidth}{0.7mm}}%
  }%
  \begin{beamercolorbox}[ht=3.25mm,dp=1.75mm]{footline}
    \usebeamerfont{footline}%
    \hypersetup{urlcolor=fg}%
    \hspace*{\beamer@leftmargin}%
%    \end{macrocode}
%
% In the left-hand box, we put the date.
%
%    \begin{macrocode}
    \makebox[0pt][l]{%
      \ifdef{\printdate}{\printdate{\insertdate}}{\insertdate}%
    }%
    \hfill
%    \end{macrocode}
%
% We use the centre box (on all slides) for the hashtag, if given.
% Otherwise we display the RDA domain name.
%
%    \begin{macrocode}
    \makebox[0pt][c]{
      \ifdefvoid{\inserthashtag}{%
        \href{https://rd-alliance.org/}{rd-alliance.org}%
      }{%
        \inserthashtag
      }\strut\par
    }%
    \hfill
%    \end{macrocode}
%
% We use the right hand box for the progress indicator.
%
%    \begin{macrocode}
    \makebox[0pt][r]{%
      \usebeamertemplate*{progress}%
    }%
    \hspace*{\beamer@rightmargin}\par
  \end{beamercolorbox}
}
%    \end{macrocode}
%
% Note that, unlike the \bthm{RDA} theme, this template does not display
% the title graphic or licence logo, so we will have to fix that later
% (see \ref{sec:titlepage}).
%
% \iffalse
%</theme2>
%<*theme3>
% \fi
%
% In the \bthm{RDA2020} theme, the footline is the same on all slides, and
% consists of a brown strip with various metadata spaced out.
%
%    \begin{macrocode}
\setbeamertemplate{footline}{%
  \begin{beamercolorbox}[ht=4.35mm,dp=2.22mm]{footline}
    \usebeamerfont{footline}%
    \hypersetup{urlcolor=fg}%
    \hspace*{\beamer@leftmargin}%
%    \end{macrocode}
%
% First, we put the date.
%
%    \begin{macrocode}
    \ifdef{\printdate}{\printdate{\insertdate}}{\insertdate}%
    \hspace*{\stretch{3}}%
%    \end{macrocode}
%
% We display the hashtag, if given. Otherwise we display the RDA domain
% name.
%
%    \begin{macrocode}
    \ifdefvoid{\inserthashtag}{%
      \adjustimage{%
        totalheight=\baselineskip,
        raise=-.2\baselineskip
      }{rda-link-white}\space
      \href{https://rd-alliance.org/}{rd-alliance.org}%
    }{%
      \inserthashtag
    }\hspace*{\stretch{3}}%
%    \end{macrocode}
%
% We display the official Twitter handles.
%
%    \begin{macrocode}
    \adjustimage{%
      totalheight=\baselineskip,
      raise=-.2\baselineskip
    }{rda-twitter-white}\space
    \href{https://twitter.com/resdatall}{@resdatall}%
    \expandafter\forcsvlist\expandafter\RDAFormatHandle\handlecsv
    \hspace*{\stretch{2}}%
%    \end{macrocode}
%
% Last, we put the progress indicator and licence logo.
%
%    \begin{macrocode}
    \usebeamertemplate*{progress}%
    \hspace*{\dimexpr\beamer@rightmargin - 0.333em\relax}%
    \ifdefvoid{\licenseLogo}{}{%
      \adjustbox{width=11.7mm,raise=-0.25\height}{\licenseLogo}%
    }%
    \hspace*{0.333em}\par
  \end{beamercolorbox}
}
%    \end{macrocode}
%
% Again, this template does not display the title graphic,
% so we will have to fix that later (see \ref{sec:titlepage}).
%
% \iffalse
%</theme3>
%<*theme4>
% \fi
%
% In the \bthm{RDA2024} theme, we set the footline to cover the footer block in
% the background images. It is already quite busy in most of them, so we are
% working with a sprung box covering the stripe on left-biased images,
% a sprung box covering the stripe on right-biased images, and a
% centre-aligned box covering the remaining space in the middle.
%
%    \begin{macrocode}
\setbeamertemplate{footline}{%
%    \end{macrocode}
%
% \begin{macro}{footLThook}
% \begin{macro}{footLHShook}
% We only use the left-hand box on the title slide (where it is fairly clear),
% for the \pkg{beamer} title graphic at the left end. In case authors need it,
% we have hooks for inserting content at the right end, \cs{footLThook} for
% title slides and \cs{footLHShook} on regular slides.
%
%    \begin{macrocode}
  \hypersetup{urlcolor=fg}%
  \begin{adjustbox}%
  {minipage=[b][0.05\paperheight][b]{0.316\paperwidth}
  ,margin={0.029\paperwidth} {0.039\paperheight} 0pt 0pt
  ,left=0pt
  }\raggedleft\strut
    \iftoggle{titlepage}{%
      \adjustbox{raise={\dimexpr-4.5mm+1em\relax}{0pt}{0pt}}{
        \usebeamercolor[fg]{titlegraphic}%
        \inserttitlegraphic
      }\hfill
      \ifdefvoid{\footLThook}{}{\footLThook}%
    }{%
      \ifdefvoid{\footLHShook}{}{\footLHShook}%
    }\strut
  \end{adjustbox}%
  \hfill
%    \end{macrocode}
% \end{macro}
% \end{macro}
%
% \begin{macro}{footCThook}
% We use the centre box (on non-title slides) for the progress meter and the
% hashtag, if given. We provide a hook, \cs{footCThook}, for inserting content
% there on title slides (but not finale slides: that space is taken).
%
%    \begin{macrocode}
  \begin{adjustbox}%
  {minipage=[b][0.05\paperheight][b]{0.3\paperwidth}
  ,margin=0pt {0.039\paperheight} 0pt 0pt
  }\centering\strut
    \iftoggle{titlepage}{%
      \iftoggle{finalepage}{}{%
        \ifdefvoid{\footCThook}{}{\footCThook}}%
    }{%
      \usebeamertemplate*{progress}\\
      \ifdefvoid{\inserthashtag}{}{\inserthashtag}%
    }\strut
  \end{adjustbox}%
  \hfill
%    \end{macrocode}
% \end{macro}
%
% \begin{macro}{footRThook}
% \begin{macro}{footRHShook}
% We only use the right hand box ib the title slide, for the licence image at
% the right end. Again, there are hooks for inserting content at the left end,
% \cs{footRThook} for title slides and \cs{footRHShook} on regular slides.
%
%    \begin{macrocode}
  \begin{adjustbox}%
  {minipage=[b][0.05\paperheight][b]{0.316\paperwidth}
  ,margin=0pt {0.039\paperheight} {0.029\paperwidth} 0pt
  ,right=0pt
  }\strut
    \ifdefvoid{\footRThook}{}{\footRThook}%
    \iftoggle{titlepage}{%
      \hfill
      \adjustbox{width=18mm,set vsize={0pt}{0pt}}{%
        \ifdefvoid{\licenseLogo}{}{%
          \ifdefvoid{\licenseUrl}{\licenseLogo}{\href{\licenseUrl}{\licenseLogo}}%
        }%
      }%
    }{%
      \ifdefvoid{\footRHShook}{}{\footRHShook}%
    }\strut
  \end{adjustbox}%
}
%    \end{macrocode}
% \end{macro}
% \end{macro}
%
% \iffalse
%</theme4>
%<*theme|theme2|theme3|theme4>
% \fi
%
% Across all themes, we default to omitting the tell-tale but seldom used
% \pkg{beamer} navigation symbols.
%
%    \begin{macrocode}
\setbeamertemplate{navigation symbols}{}

%    \end{macrocode}
%
% \subsection{Theme: Presentation inner theme}\label{sec:inner}
%
% Here we style lists and the \cs{maketitle} command.
%
% \subsubsection{Lists}
%
% \iffalse
%</theme|theme2|theme3|theme4>
%<*theme>
% \fi
%
% In the \bthm{RDA} theme, itemized lists use square markers at all levels.
%
%    \begin{macrocode}
\setbeamertemplate{itemize items}[square]
%    \end{macrocode}
%
% \iffalse
%</theme>
%<*theme2|theme4>
% \fi
% In the \bthm{RDA2016} and \bthm{RDA2024} themes, itemized lists use chevrons
% at the top level, en dashes at the second, and circles at the third.
%
%    \begin{macrocode}
\AtEndPreamble{\RequirePackage{fontawesome}}
\defbeamertemplate*{itemize item}{chevron}{\faAngleRight}
\defbeamertemplate*{itemize subitem}{dash}{\textbf{\textendash}}
\setbeamertemplate{itemize subsubitem}[circle]
%    \end{macrocode}
%
% \iffalse
%</theme2|theme4>
%<*theme3>
% \fi
% In the \bthm{RDA2020} theme, itemized lists use little RDA logos at
% the top level, chevrons at the second, and en dashes at the third.
%
%    \begin{macrocode}
\AtEndPreamble{\RequirePackage{fontawesome}}
\defbeamertemplate*{itemize item}{rda}{%
  \adjustimage{height=1em,raise=-.2em}{rda-bullet}}
\defbeamertemplate*{itemize subitem}{chevron}{\faAngleRight}
\defbeamertemplate*{itemize subsubitem}{dash}{\textbf{\textendash}}
%    \end{macrocode}
%
% \iffalse
%</theme3>
%<*theme|theme2|theme3|theme4>
% \fi
%
% \subsubsection{Title page template}
% \label{sec:titlepage}
%
% Since space is tight on most iterations of the title slide, we change the
% default presentation of multiple institutions so they can share a line.
%
%    \begin{macrocode}
\mode<presentation>{\def\beamer@andinst{\quad}}
%    \end{macrocode}
%
% \iffalse
%</theme|theme2|theme3|theme4>
%<*theme>
% \fi
%
% The \pkg{beamer} title page template lays out the elements of the title page.
% We need a different one for each theme.
%
% For the \bthm{RDA} theme, we begin by setting the special background image.
%
%    \begin{macrocode}
\setbeamertemplate{title page}{%
  \global\toggletrue{titlepage}%
  \setkeys{beamerframe}{background=rda-bg-title1}%
%    \end{macrocode}
%
% The background image is in thirds, with the majority of the titling content
% expected to go in the middle third. So we start by blocking out the top
% third (about 24\,mm) with a \cs{vbox}.
%
%    \begin{macrocode}
  \vbox to 24mm{}%
%    \end{macrocode}
%
% The middle third is 32\,mm high, so we model it as a fixed-size \env{minipage}.
% Into this we put the title, author and institute lines. We use stretchy skips
% to ensure the elements are evenly spaced.
%
%    \begin{macrocode}
  \begin{minipage}[b][32mm][c]{\textwidth}
    \vspace*{\stretch{1}}%
    \begin{beamercolorbox}[sep=0pt,center]{title}
      \usebeamerfont{title}\inserttitle\par%
      \ifdefvoid{\insertsubtitle}{}{%
        \vskip0.25em%
        {\usebeamerfont{subtitle}\usebeamercolor[fg]{subtitle}\insertsubtitle\par}%
      }%
    \end{beamercolorbox}%
    \vspace{\stretch{1}}%
    \begin{beamercolorbox}[sep=0pt,center]{author}
      \usebeamerfont{author}\insertauthor
    \end{beamercolorbox}
    \vspace{\stretch{1}}%
    \begin{beamercolorbox}[sep=0pt,center]{institute}
      \usebeamerfont{institute}\insertinstitute
    \end{beamercolorbox}
    \vspace*{\stretch{1}}%
  \end{minipage}
  \par\vspace*{2mm}
%    \end{macrocode}
%
% The lower third contains the strapline and domain name, but there is about
% 10\,mm available in which to squeeze the occasion and date. We model this once
% more as a fixed-size \env{minipage}.
%
%    \begin{macrocode}
  \begin{minipage}[b][10mm][c]{\textwidth}
    \begin{beamercolorbox}[sep=0pt,center]{date}
      \usebeamerfont{date}%
      \ifdefvoid{\insertoccasion}{}{\insertoccasion\\}%
      \ifdef{\printdate}{\printdate{\insertdate}}{\insertdate}%
    \end{beamercolorbox}
  \end{minipage}
  \vspace*{2mm}
}
%    \end{macrocode}
%
% \iffalse
%</theme>
%<*theme2>
% \fi
%
% There is more freedom available with the \bthm{RDA2016} theme. Again, we start
% by reserving space for the logo inserted by the \texttt{headline} template
% (see \ref{sec:headline}).
%
%    \begin{macrocode}
\setbeamertemplate{title page}{%
  \global\toggletrue{titlepage}%
  \setkeys{beamerframe}{background=rda-bg-title1}%
  \vbox to 33.8mm{}%
%    \end{macrocode}
%
% We then provide the metadata elements, with stretchy skips top and bottom to
% centre them on the page.
%
%    \begin{macrocode}
  \vfill
  \begin{beamercolorbox}[sep=0pt,center]{title}
    \usebeamerfont{title}\inserttitle\par%
    \ifdefvoid{\insertsubtitle}{}{%
      \vskip0.25em%
      {\usebeamerfont{subtitle}\usebeamercolor[fg]{subtitle}\insertsubtitle\par}%
    }%
  \end{beamercolorbox}%
  \bigskip
  \begin{beamercolorbox}[sep=0pt,center]{author}
    \usebeamerfont{author}\insertauthor
  \end{beamercolorbox}
  \smallskip
  \begin{beamercolorbox}[sep=0pt,center]{institute}
    \usebeamerfont{institute}\insertinstitute
  \end{beamercolorbox}
  \bigskip\medskip
  \begin{beamercolorbox}[sep=0pt,center]{date}
    \usebeamerfont{date}%
    \ifdefvoid{\insertoccasion}{}{\insertoccasion\\}%
    \ifdef{\printdate}{\printdate{\insertdate}}{\insertdate}%
  \end{beamercolorbox}
  \vfill
%    \end{macrocode}
%
% Finally, we provide an extra zero-height line for the title graphic and
% licence logo.
%
%    \begin{macrocode}
  \raisebox{0pt}[0pt][0pt]{\makebox[\linewidth]{%
    \usebeamercolor[fg]{titlegraphic}%
    \inserttitlegraphic\hfill
    \parbox[b]{18mm}{%
      \ifdefvoid{\licenseLogo}{\hbox{}}{%
        \ifdefvoid{\licenseUrl}{\licenseLogo}{\href{\licenseUrl}{\licenseLogo}}%
      }%
    }%
  }}%
}
%    \end{macrocode}
%
% \iffalse
%</theme2>
%<*theme3>
% \fi
%
% The \bthm{RDA2020} theme has the same small logo at the top left as all other
% slides, so we reflect this in the layout.
%
%    \begin{macrocode}
\setbeamertemplate{title page}{%
  \global\toggletrue{titlepage}%
  \vspace*{12mm}%
%    \end{macrocode}
%
% We then provide the metadata elements, with stretchy skips top and bottom to
% centre them on the page, but fixed skips inbetween.
%
%    \begin{macrocode}
  \vfill
  \begin{beamercolorbox}[sep=0pt,center]{title}
    \usebeamerfont{title}\inserttitle\par%
    \ifdefvoid{\insertsubtitle}{}{%
      \vskip0.25em%
      {\usebeamerfont{subtitle}\usebeamercolor[fg]{subtitle}\insertsubtitle\par}%
    }%
  \end{beamercolorbox}%
  \bigskip
  \begin{beamercolorbox}[sep=0pt,center]{author}
    \usebeamerfont{author}\insertauthor
  \end{beamercolorbox}
  \smallskip
  \begin{beamercolorbox}[sep=0pt,center]{institute}
    \usebeamerfont{institute}\insertinstitute
  \end{beamercolorbox}
  \bigskip\medskip
  \begin{beamercolorbox}[sep=0pt,center]{date}
    \usebeamerfont{date}%
    \ifdefvoid{\insertoccasion}{}{\insertoccasion\\}%
    \ifdef{\printdate}{\printdate{\insertdate}}{\insertdate}%
  \end{beamercolorbox}
  \vfill
%    \end{macrocode}
%
% Finally, we provide an extra zero-height line for the title graphic.
%
%    \begin{macrocode}
  \raisebox{0pt}[0pt][0pt]{\makebox[\linewidth]{%
    \usebeamercolor[fg]{titlegraphic}%
    \inserttitlegraphic
  }}%
}

%    \end{macrocode}
%
% \iffalse
%</theme3>
%<*theme4>
% \fi
%
% \begin{macro}{titleTLhook}
% The \bthm{RDA2024} theme's title slide has several bold elements to work
% around. In the top half, the middle third is the logo, and we've used the
% right hand side in the \texttt{headline} for RDA group information, but the
% left hand side is available. We supply a hook for putting stuff there. Text
% should align with RDA group information if supplied
%
%    \begin{macrocode}
\RequirePackage{abspos}
\setbeamertemplate{title page}{%
  \global\toggletrue{titlepage}%
  \setkeys{beamerframe}{background=rda24-title}%
  \absput{%
    \begin{adjustbox}%
    { minipage=[b][0.87\paperheight]{0.94\paperwidth}
    , margin=0pt {0.05\paperheight} 0pt 0pt
    }
      \begin{adjustbox}%
      { minipage=[b][0.25\paperheight][t]{0.292\paperwidth}
      , margin=0pt 0pt {0.03\paperwidth} {0.095\paperheight}
      }
        \ifdefvoid{\titleTLhook}{}{\titleTLhook}%
      \end{adjustbox}\vfill
%    \end{macrocode}
% \end{macro}
%
% In the lower half, we have a very narrow clear strip into which to cram the
% title, author and institute lines. We use stretchy skips to space the elements
% evenly.
%
% The hills at the bottom left and right make an interesting backdrop for
% occasion and date, but if the occasion is missing, there should be enough
% space for the date in the middle valley.
%
%    \begin{macrocode}
      \begin{adjustbox}%
      { minipage=[b][0.37\paperheight][c]{\textwidth}
      , raise={-0.02\paperheight}{0.35\paperheight}{0pt}
      }
        \vspace*{\stretch{3}}%
        \begin{beamercolorbox}[sep=0pt,center]{title}
          \usebeamerfont{title}\inserttitle\par%
          \ifdefvoid{\insertsubtitle}{}{\vskip0.25em{%
            \usebeamerfont{subtitle}\usebeamercolor[fg]{subtitle}%
            \insertsubtitle\par
          }}%
        \end{beamercolorbox}%
        \vspace{\stretch{2}}%
        \begin{beamercolorbox}[sep=0pt,center]{author}
          \usebeamerfont{author}\insertauthor
        \end{beamercolorbox}
        \begin{beamercolorbox}[sep=0pt,center]{institute}
          \usebeamerfont{institute}\insertinstitute
        \end{beamercolorbox}
        \vspace{\stretch{2}}%
        \begin{beamercolorbox}[sep=0pt,center]{date}
          \ifdefvoid{\insertoccasion}{%
            \usebeamercolor[fg]{date}\usebeamerfont{date}%
            \ifdef{\printdate}{\printdate{\insertdate}}{\insertdate}%
          }{}\strut
        \end{beamercolorbox}
      \end{adjustbox}\\
      \begin{adjustbox}%
      { minipage=[b][0.15\paperheight][t]{0.36\paperwidth}
      }\centering
        \ifdefvoid{\insertoccasion}{}{%
          \usebeamercolor[fg]{date}\usebeamerfont{date}\insertoccasion}%
      \end{adjustbox}\hfill
      \begin{adjustbox}%
      { minipage=[b][0.15\paperheight][t]{0.36\paperwidth}
      }\centering
        \ifdefvoid{\insertoccasion}{}{%
          \usebeamercolor[fg]{negdate}\usebeamerfont{negdate}%
          \ifdef{\printdate}{\printdate{\insertdate}}{\insertdate}%
        }%
      \end{adjustbox}%
    \end{adjustbox}
  }%
}
%    \end{macrocode}
%
% \iffalse
%</theme4>
%<*theme|theme2|theme3|theme4>
% \fi
%
% \subsubsection{Finale page}
%
% We provide a \cs{finale} command for use on the last slide, to parallel
% \cs{maketitle}. Here is the version for presentations.
%
% \begin{macro}{finale}
% \begin{macro}{rdaslides@finale@text}
% In presentation mode, \cs{finale} is intended to be used in a frame. It prints
% a thank you and some key information that the audience can note down during
% the questions. To parallel \cs{maketitle}, it triggers the use of a
% \pkg{beamer} template and the \texttt{titlepage} layout.
%
%    \begin{macrocode}
\mode<presentation>{%
  \newcommand{\finale}[1][\empty]{%
    \ifstrempty{#1}{}{\def\rdaslides@finale@text{#1}}%
    \usebeamertemplate{finale page}%
    \let\rdaslides@finale@text=\relax
  }
}
%    \end{macrocode}
% \end{macro}
% \end{macro}
%
% \iffalse
%</theme|theme2|theme3|theme4>
%<*theme>
% \fi
%
% Here is the template for the \bthm{RDA} theme. This time we use the
% alternative title background. We insert a spacer to represent the top third.
%
%    \begin{macrocode}
\setbeamertemplate{finale page}{%
  \global\toggletrue{titlepage}\global\toggletrue{finalepage}%
  \setkeys{beamerframe}{background=rda-bg-title2}%
  \vbox to 24mm{}%
%    \end{macrocode}
%
% The background image again leaves us with the middle third of the slide
% (32\,mm) into which to cram some text. Again, we use a fixed-size
% \env{minipage} for this, and use stretchy skips to ensure the elements are
% evenly spaced. We start with a thank you.
%
%    \begin{macrocode}
  \begin{minipage}[b][32mm][s]{\textwidth}
    \vspace*{\stretch{1}}%
    \begin{beamercolorbox}[sep=0pt,center]{title}
      \usebeamerfont{title}Thank you for your attention\par%
    \end{beamercolorbox}%
%    \end{macrocode}
%
% If the author URL has been provided, we display that.
%
%    \begin{macrocode}
    \ifdefvoid{\insertauthorurl}{}{%
      \vspace{\stretch{1}}%
      \begin{beamercolorbox}[sep=0pt,center]{institute}
        \usebeamerfont{institute}\insertauthor: \url{\insertauthorurl}\par%
      \end{beamercolorbox}%
    }%
%    \end{macrocode}
%
% If the RDA group URL has been provided, we display that.
%
%    \begin{macrocode}
    \ifdefvoid{\rdaGroupUrl}{}{%
      \vspace{\stretch{1}}%
      \begin{beamercolorbox}[sep=0pt,center]{institute}
        \usebeamerfont{institute}%
        \ifdefvoid{\rdaGroupName}{}{\rdaGroupName
          \ifdefvoid{\rdaGroupType}{}{ \rdaGroupType}:
        }\url{\rdaGroupUrl}\par%
      \end{beamercolorbox}%
    }%
%    \end{macrocode}
%
% Lastly, we display whatever is in the optional argument, if one has been
% provided.
%
%    \begin{macrocode}
    \ifdefvoid{\rdaslides@finale@text}{}{%
      \vspace{\stretch{1}}%
      \begin{beamercolorbox}[sep=0pt,center]{institute}
        \usebeamerfont{institute}%
        \rdaslides@finale@text\par%
      \end{beamercolorbox}%
    }
    \vspace*{\stretch{1}}%
  \end{minipage}
  \vspace*{15mm}
}
%    \end{macrocode}
%
% \iffalse
%</theme>
%<*theme2>
% \fi
%
% Here is the corresponding template for the \bthm{RDA2016} theme. We insert a
% spacer to protect the logo at the top of the page.
%
%    \begin{macrocode}
\setbeamertemplate{finale page}{%
  \global\toggletrue{titlepage}\global\toggletrue{finalepage}%
  \vbox to 33.8mm{}%
%    \end{macrocode}
%
% We continue with a thank you.
%
%    \begin{macrocode}
  \vspace*{\stretch{1}}%
  \begin{beamercolorbox}[sep=0pt,center]{title}
    \usebeamerfont{title}Thank you for your attention\par%
  \end{beamercolorbox}%
%    \end{macrocode}
%
% If the author URL has been provided, we display that.
%
%    \begin{macrocode}
  \ifdefvoid{\insertauthorurl}{}{%
    \bigskip
    \begin{beamercolorbox}[sep=0pt,center]{author}
      \usebeamerfont{author}\insertauthor: \url{\insertauthorurl}\par%
    \end{beamercolorbox}%
  }%
%    \end{macrocode}
%
% If the RDA group URL has been provided, we display that.
%
%    \begin{macrocode}
  \ifdefvoid{\rdaGroupUrl}{}{%
    \bigskip
    \begin{beamercolorbox}[sep=0pt,center]{author}
      \usebeamerfont{author}%
      \ifdefvoid{\rdaGroupName}{}{\rdaGroupName
        \ifdefvoid{\rdaGroupType}{}{ \rdaGroupType}:
      }\url{\rdaGroupUrl}\par%
    \end{beamercolorbox}%
  }%
%    \end{macrocode}
%
% Lastly, we display whatever is in the optional argument, if one has been
% provided.
%
%    \begin{macrocode}
  \ifdefvoid{\rdaslides@finale@text}{}{%
    \bigskip
    \begin{beamercolorbox}[sep=0pt,center]{date}
      \usebeamerfont{date}%
      \rdaslides@finale@text\par%
    \end{beamercolorbox}%
  }
  \vspace*{\stretch{1}}%
}
%    \end{macrocode}
%
% \iffalse
%</theme2>
%<*theme3>
% \fi
%
% Here is the corresponding template for the \bthm{RDA2016} theme. We insert a
% spacer to protect the logo at the top of the page.
%
%    \begin{macrocode}
\setbeamertemplate{finale page}{%
  \global\toggletrue{titlepage}\global\toggletrue{finalepage}%
  \vspace*{12mm}%
%    \end{macrocode}
%
% We continue with a thank you.
%
%    \begin{macrocode}
  \vspace*{\stretch{1}}%
  \begin{beamercolorbox}[sep=0pt,center]{title}
    \usebeamerfont{title}Thank you for your attention\par%
  \end{beamercolorbox}%
%    \end{macrocode}
%
% If the author URL has been provided, we display that.
%
%    \begin{macrocode}
  \ifdefvoid{\insertauthorurl}{}{%
    \bigskip
    \begin{beamercolorbox}[sep=0pt,center]{author}
      \usebeamerfont{author}\insertauthor: \url{\insertauthorurl}\par%
    \end{beamercolorbox}%
  }%
%    \end{macrocode}
%
% If the RDA group URL has been provided, we display that.
%
%    \begin{macrocode}
  \ifdefvoid{\rdaGroupUrl}{}{%
    \bigskip
    \begin{beamercolorbox}[sep=0pt,center]{author}
      \usebeamerfont{author}%
      \ifdefvoid{\rdaGroupName}{}{\rdaGroupName
        \ifdefvoid{\rdaGroupType}{}{ \rdaGroupType}:
      }\url{\rdaGroupUrl}\par%
    \end{beamercolorbox}%
  }%
%    \end{macrocode}
%
% Lastly, we display whatever is in the optional argument, if one has been
% provided.
%
%    \begin{macrocode}
  \ifdefvoid{\rdaslides@finale@text}{}{%
    \bigskip
    \begin{beamercolorbox}[sep=0pt,center]{date}
      \usebeamerfont{date}%
      \rdaslides@finale@text\par%
    \end{beamercolorbox}%
  }
  \vspace*{\stretch{1}}%
}
%    \end{macrocode}
%
% \iffalse
%</theme3>
%<*theme4>
% \fi
%
% Here is the template for the \bthm{RDA2024} theme. We use the dedicated
% finale page background.
%
%    \begin{macrocode}
\setbeamertemplate{finale page}{%
  \global\toggletrue{titlepage}\global\toggletrue{finalepage}%
  \setkeys{beamerframe}{background=rda24-finale}%
  \vbox to 10mm {}\vfill
%    \end{macrocode}
%
% There is about 58\,mm clear vertical space into which to fit the elements.
% We use a fixed-size \env{minipage} for this, and use stretchy skips
% to ensure the elements are evenly spaced. We start with a thank you,
% approximating the design used in the PowerPoint template.
%
%    \begin{macrocode}
  \begin{adjustbox}{minipage=[c][58mm]{\textwidth}}%
    \begin{beamercolorbox}[sep=3mm,center]{thanks}
      \usebeamerfont{thanks}%
      \rule[0.3em]{3em}{2pt}\enskip THANK YOU\enskip\rule[0.3em]{3em}{2pt}\par%
    \end{beamercolorbox}%
%    \end{macrocode}
%
% If the author URL has been provided, we display that.
%
%    \begin{macrocode}
    \ifdefvoid{\insertauthorurl}{}{%
      \bigskip%
      \begin{beamercolorbox}[sep=0pt,center]{author}
        \usebeamerfont{institute}\insertauthor: \url{\insertauthorurl}\par%
      \end{beamercolorbox}%
    }%
%    \end{macrocode}
%
% If the RDA group URL has been provided, we display that.
%
%    \begin{macrocode}
    \ifdefvoid{\rdaGroupUrl}{}{%
      \bigskip%
      \begin{beamercolorbox}[sep=0pt,center]{author}
        \usebeamerfont{institute}%
        \ifdefvoid{\rdaGroupName}{}{\rdaGroupName
          \ifdefvoid{\rdaGroupType}{}{ \rdaGroupType}:\\
        }\url{\rdaGroupUrl}\par%
      \end{beamercolorbox}%
    }%
%    \end{macrocode}
%
% Lastly, we display whatever is in the optional argument, if one has been
% provided.
%
%    \begin{macrocode}
    \ifdefvoid{\rdaslides@finale@text}{}{%
      \bigskip%
      \begin{beamercolorbox}[sep=0pt,center]{author}
        \usebeamerfont{institute}%
        \rdaslides@finale@text\par%
      \end{beamercolorbox}%
    }
  \end{adjustbox}
}
%    \end{macrocode}
%
% \iffalse
%</theme4>
%<*class>
% \fi
%
% \subsection{Class: Endmatter}
%
% \begin{macro}{finale}
% In article mode, we will present relevant information laid out in tabular
% fashion with \pkg{tabularx}. (We could do it with minipages but this is
% fractionally easier.)
%
%    \begin{macrocode}
\mode<article>{
  \RequirePackage{tabularx,ifpdf}
%    \end{macrocode}
%
% We begin with a horizontal rule.
%
%    \begin{macrocode}
  \newcommand{\finale}[1][\empty]{%
    \vbox{}
    \begin{small}
      \rule[1em]{\textwidth}{0.8pt}\par
%    \end{macrocode}
%
% We then draw a table with logos on the left and corresponding text on the
% right.
%
%    \begin{macrocode}
      \setlength{\extrarowheight}{1ex}%
      \renewcommand{\tabularxcolumn}[1]{m{##1}}
      \begin{tabularx}{\textwidth}{@{}m{22mm}X@{}}
%    \end{macrocode}
%
% If a licence statement has been provided, we add a row displaying it.
%
%    \begin{macrocode}
      \ifdefvoid{\licenseStatement}{}{%
        \ifdefvoid{\licenseLogo}{}{\parbox[c]{\hsize}{\licenseLogo}} &
        \licenseStatement
        \ifdefvoid{\licenseUrl}{}{: \url{\licenseUrl}} \\
      }%
%    \end{macrocode}
%
% We then add a row about the RDA. If a group logo has been defined we display
% it. Otherwise, we display the RDA logo.
%
%    \begin{macrocode}
      \ifdefvoid{\rdaGroupLogo}{%
        \includegraphics[width=\hsize]{rda-logo}%
      }{%
        \parbox[c]{\hsize}{\rdaGroupLogo}%
      } &
%    \end{macrocode}
%
% If group information has not been defined, we explain what the RDA is.
%
%    \begin{macrocode}
      \ifdefvoid{\rdaGroupName}{%
        The Research Data Alliance is supported by the European Commission,
        the National Science Foundation and other U.S. agencies,
        and the Australian Government.\par
        \vspace{1ex}%
        For more information, please visit \url{https://rd-alliance.org/}%
%    \end{macrocode}
%
% Otherwise we explain about the group.
%
%    \begin{macrocode}
      }{%
        This work was developed as part of the Research Data Alliance (RDA)
        \rdaGroupName\ifdefvoid{\rdaGroupType}{}{ \rdaGroupType}, and we
        acknowledge the support provided by the RDA community and structures.
        \ifdefvoid{\rdaGroupUrl}{}{%
          \par\vspace{1ex}%
          For more information, please visit \url{\rdaGroupUrl}%
        }%
      }\\
      \end{tabularx}
%    \end{macrocode}
%
% Lastly, if the optional argument has been provided, we typeset it below the
% table.
%
%    \begin{macrocode}
      \ifx\empty#1\else\par\vskip2pt #1\par\fi
    \end{small}
  }
}
%    \end{macrocode}
% \end{macro}
%
% \subsection{Class: Loading the beamer theme}
%
% For styling the presentation, we use the accompanying \pkg{beamer} theme.
%
%    \begin{macrocode}
\usetheme{\rdaslides@theme}
%    \end{macrocode}
%
% \iffalse
%</class>
%<*palette>
% \fi
%
% \subsection{Research Data Alliance colour palette}
% \label{sec:rdacolors}
%
% These settings are separated out into \texttt{rdacolors.sty}.
%
% The RDA colour palette consists of four main colours:
% Pantone 478 \sample[0.7em]{rdabrown},
% Pantone 369 \sample[0.7em]{rdagreen},
% Pantone 3965 \sample[0.7em]{rdayellow}, and
% Pantone Cool Gray 7 \sample[0.7em]{rdagrey}.
%
%    \begin{macrocode}
\RequirePackage{xcolor}

\xdefinecolor{rdabrown}{RGB}{107,47,33}%         Pantone 478
\xdefinecolor{rdagreen}{RGB}{85,177,71}%         Pantone 369
\xdefinecolor{rdayellow}{RGB}{245,235,6}%       Pantone 3965
\xdefinecolor{rdagrey}{RGB}{144,149,157}%        Pantone Cool Gray 7

%    \end{macrocode}
%
% As variants of these, the palette also contains the same colours at 50\% and
% 25\% saturation.
%
%    \begin{macrocode}
\xdefinecolor{rdamidbrown}{RGB}{171,128,115}%    Pantone 478 at 50%
\xdefinecolor{rdamidgreen}{RGB}{191,221,149}%    Pantone 369 at 50%
\xdefinecolor{rdamidyellow}{RGB}{253,236,133}%   Pantone 3965 at 50%
\xdefinecolor{rdamidgrey}{RGB}{192,195,199}%     Pantone Cool Gray 7 at 50%

\xdefinecolor{rdalightbrown}{RGB}{208,183,173}%  Pantone 478 at 25%
\xdefinecolor{rdalightgreen}{RGB}{224,237,201}%  Pantone 369 at 25%
\xdefinecolor{rdalightyellow}{RGB}{255,245,194}% Pantone 3965 at 25%
\xdefinecolor{rdalightgrey}{RGB}{219,220,223}%   Pantone Cool Gray 7 at 25%

%    \end{macrocode}
%
% We also provide some handy aliases
%
%    \begin{macrocode}
\colorlet{rdagray}{rdagrey}
\colorlet{rdamidgray}{rdamidgrey}
\colorlet{rdalightgray}{rdalightgrey}
\colorlet{warm}{rdalightyellow}
\colorlet{cool}{rdalightgrey}
%    \end{macrocode}
%
% \iffalse
%</palette>
%<*mscwg>
% \fi
%
% \subsection{Sample RDA group package: MSCWG}
% \label{sec:rdamscwg}
%
% \subsubsection{Group logo}
%
% For perfect fidelity we draw the logo with \pkg{tikz} rather than use a bitmap.
% This means we need to ensure it is loaded, along with \pkg{keyval} for option
% handling, \pkg{ifthen} for implementing the option handling logic, and \pkg{calc} for
% calculating lengths. Note that the Metadata Standards Catalog Working Group
% recycles the logo of the Metadata Standards Directory Working Group, hence
% the macros below use \texttt{msd} instead of \texttt{msc}.
%
%    \begin{macrocode}
\RequirePackage{ifthen,calc,keyval}
\RequirePackage{tikz}
\usetikzlibrary{shapes.geometric,positioning}

%    \end{macrocode}
%
% We load the RDA colour palette, if not already available.
%
%    \begin{macrocode}
\usepackage{rdacolors}

%    \end{macrocode}
%
% \begin{macro}{msdwgl@unit}
% \begin{macro}{msdwgl@unit@calc}
% We provide options for scaling the logo without changing the aspect ratio.
% Two lengths are needed to scale the logo:
% \begin{itemize}
% \item \cs{msdwgl@unit} is the actual scale factor;
% \item \cs{msdwgl@unit@calc} is the scale factor calculated from the options.
% \end{itemize}
%
%    \begin{macrocode}
\newlength{\msdwgl@unit}
\newlength{\msdwgl@unit@calc}
%    \end{macrocode}
% \end{macro}
% \end{macro}
%
% \begin{optionkey}{height}
% The \key{height} option sets the maximum height for the logo.
% With \cs{msdwgl@unit} set to 1pt, the logo ends up 59.690pt high. So we
% calculate what \cs{msdwgl@unit} would need to be to achieve the target height.
% We store this in \cs{msdwgl@unit@calc} unless \cs{msdwgl@unit@calc} is a shorter
% (but non-zero) length.
%
%    \begin{macrocode}
\define@key{msdwgl}{height}{%
  \setlength{\@tempdima}{#1}%
  \setlength{\@tempdimb}{\@tempdima / \real{59.690}}%
  \ifthenelse{%
    \lengthtest{\msdwgl@unit@calc = 0pt}\OR
    \lengthtest{\@tempdimb < \msdwgl@unit@calc}%
  }{%
    \setlength{\msdwgl@unit@calc}{\@tempdimb}%
  }{}%
}
%    \end{macrocode}
% \end{optionkey}
%
% \begin{optionkey}{width}
% The \key{width} option sets the maximum width for the logo.
% With \cs{msdwgl@unit} set to 1pt, the logo ends up 62.091pt wide. So we
% calculate what \cs{msdwgl@unit} would need to be to achieve the target width.
% We store this in \cs{msdwgl@unit@calc} unless \cs{msdwgl@unit@calc} is a shorter
% (but non-zero) length.
%
%    \begin{macrocode}
\define@key{msdwgl}{width}{%
  \setlength{\@tempdima}{#1}%
  \setlength{\@tempdimb}{\@tempdima / \real{62.091}}%
  \ifthenelse{%
    \lengthtest{\msdwgl@unit@calc = 0pt}\OR
    \lengthtest{\@tempdimb < \msdwgl@unit@calc}%
  }{%
    \setlength{\msdwgl@unit@calc}{\@tempdimb}%
  }{}%
}
%    \end{macrocode}
% \end{optionkey}
%
% \begin{optionkey}{scale}
% We also provide the option \key{scale} for unconditionally setting the scale to a
% given numeric factor, where 1 means \cs{msdwgl@unit} equals 1pt.
%
%    \begin{macrocode}
\define@key{msdwgl}{scale}{%
  \setlength{\@tempdima}{1pt}%
  \setlength{\msdwgl@unit@calc}{#1\@tempdima}%
}
%    \end{macrocode}
% \end{optionkey}
%
% \begin{macro}{msdwgl@line@color}
% \begin{optionkey}{outline}
% In case the logo is put on a coloured background, we provide the option
% \key{outline} for outlining it in a given colour (white by default).
%
%    \begin{macrocode}
\newcommand{\msdwgl@line@color}{none}
\define@key{msdwgl}{outline}[white]{\renewcommand{\msdwgl@line@color}{#1}}

%    \end{macrocode}
% \end{optionkey}
% \end{macro}
%
% \begin{macro}{MSDWGLogo}
% The logo itself is drawn with the \cs{MSDWGLogo} command. The \pkg{keyval} options
% are read from the optional argument. There is no mandatory argument.
%
%    \begin{macrocode}
\newcommand{\MSDWGLogo}[1][]{%
  \bgroup
%    \end{macrocode}
%
% First we reset \cs{msdwgl@unit@calc} to 0pt, then read in the user keys.
%
%    \begin{macrocode}
  \setkeys{msdwgl}{scale=0,#1}%
%    \end{macrocode}
%
% If \cs{msdwgl@unit@calc} has been set, we scale the logo accordingly. Otherwise
% we use the default scale of \cs{msdwgl@unit} = 1pt.
%
%    \begin{macrocode}
  \ifthenelse{\lengthtest{\msdwgl@unit@calc > 0pt}}{%
    \setlength{\msdwgl@unit}{\msdwgl@unit@calc}%
  }{%
    \setlength{\msdwgl@unit}{1pt}%
  }%
%    \end{macrocode}
%
% Now we come to draw the logo. The \key{text} node is used to centre the logo
% vertically with respect to the surrounding text. The graphic itself is
% achieved as a matrix of cylinders.
%
%    \begin{macrocode}
  \begin{tikzpicture}
    [ inner sep = 0pt
    , outer sep = 0pt
    , baseline = (text.base)
    , line width = 0.4pt
    ]
  \matrix
    [ ampersand replacement = \&
    , nodes =
      { cylinder
      , cylinder uses custom fill
      , cylinder end fill=rdayellow
      , aspect=1.0
      , rotate=43
      , anchor=center
      , draw=\msdwgl@line@color
      }
    , row 1 column 1/.style = {cylinder body fill=rdagreen}
    , row 1 column 2/.style = {cylinder body fill=rdabrown}
    , row 2 column 1/.style = {cylinder body fill=rdabrown}
    , row 2 column 2/.style = {cylinder body fill=rdagreen}
    , column sep = -3\msdwgl@unit
    , row sep = -4\msdwgl@unit
    ]
    (logo)
    {
      \node{\phantom{\rule{4\msdwgl@unit}{20\msdwgl@unit}}}; \&
      \node{\phantom{\rule{4\msdwgl@unit}{20\msdwgl@unit}}}; \\
      \node{\phantom{\rule{4\msdwgl@unit}{20\msdwgl@unit}}}; \&
      \node{\phantom{\rule{4\msdwgl@unit}{20\msdwgl@unit}}}; \\
    };
  \node (text) at (logo.center) {\phantom{RDA}};
  \end{tikzpicture}%
  \egroup
}

%    \end{macrocode}
% \end{macro}
%
% \subsubsection{Group information}
%
% \begin{macro}{rdaGroupLogo}
% \begin{macro}{rdaGroupName}
% \begin{macro}{rdaGroupType}
% \begin{macro}{rdaGroupUrl}
% Finally, we define hooks that are recognized by \texttt{rdaslides.cls}.
%
%    \begin{macrocode}
\def\rdaGroupLogo{\MSDWGLogo[width=\hsize]}
\def\rdaGroupName{Metadata Standards Catalog}
\def\rdaGroupType{Working Group}
\def\rdaGroupUrl{https://rd-alliance.org/groups/metadata-standards-catalog-working-group.html}
%    \end{macrocode}
% \end{macro}
% \end{macro}
% \end{macro}
% \end{macro}
%
% \iffalse
%</mscwg>
%<*msdwg>
% \fi
%
% \subsection{Sample RDA group package: MSDWG}
% \label{sec:rdamsdwg}
%
% \begin{macro}{rdaGroupName}
% \begin{macro}{rdaGroupUrl}
% The Metadata Standards Directory Working Group has the same properties
% as the Metadata Standards Catalog Working Group except the name and URL.
%
%    \begin{macrocode}
\RequirePackage{rdamscwg}
\def\rdaGroupName{Metadata Standards Directory}
\def\rdaGroupUrl{%
  https://rd-alliance.org/groups/metadata-standards-directory-working-group.html}
%    \end{macrocode}
% \end{macro}
% \end{macro}
%
% \iffalse
%</msdwg>
%<*mig>
% \fi
%
% \subsection{Sample RDA group package: MIG}
% \label{sec:rdamig}
%
% \subsubsection{Group logo}
%
% Again, we draw the logo with \pkg{tikz} rather than use a bitmap.
% This means we need to ensure it is loaded, along with \pkg{keyval} for option
% handling, \pkg{ifthen} for implementing the option handling logic, and \pkg{calc} for
% calculating lengths.
%
%    \begin{macrocode}
\RequirePackage{ifthen,calc,keyval}
\RequirePackage{tikz}
\usetikzlibrary{calc,shapes.geometric,positioning}

%    \end{macrocode}
%
% We load the RDA colour palette, if not already available.
%
%    \begin{macrocode}
\usepackage{rdacolors}

%    \end{macrocode}
%
% \begin{macro}{migl@unit}
% \begin{macro}{migl@unit@calc}
% We provide options for scaling the logo without changing the aspect ratio.
% Two lengths are needed to scale the logo:
% \begin{itemize}
% \item \cs{migl@unit} is the actual scale factor;
% \item \cs{migl@unit@calc} is the scale factor calculated from the options.
% \end{itemize}
%
%    \begin{macrocode}
\newlength{\migl@unit}
\newlength{\migl@unit@calc}
%    \end{macrocode}
% \end{macro}
% \end{macro}
%
% \begin{optionkey}{height}
% The \key{height} option sets the maximum height for the logo.
% With \cs{migl@unit} set to 1pt, the logo ends up 50pt high. So we
% calculate what \cs{migl@unit} would need to be to achieve the target height.
% We store this in \cs{migl@unit@calc} unless \cs{migl@unit@calc} is a shorter
% (but non-zero) length.
%
%    \begin{macrocode}
  \define@key{migl}{height}{%
    \setlength{\@tempdima}{#1}%
    \setlength{\@tempdimb}{\@tempdima / \real{59.690}}%
    \ifthenelse{%
      \lengthtest{\migl@unit@calc = 0pt}\OR
      \lengthtest{\@tempdimb < \migl@unit@calc}%
    }{%
      \setlength{\migl@unit@calc}{\@tempdimb}%
    }{}%
  }
%    \end{macrocode}
% \end{optionkey}
%
% \begin{optionkey}{width}
% The \key{width} option sets the maximum width for the logo.
% With \cs{migl@unit} set to 1pt, the logo ends up 60pt wide. So we
% calculate what \cs{migl@unit} would need to be to achieve the target width.
% We store this in \cs{migl@unit@calc} unless \cs{migl@unit@calc} is a shorter
% (but non-zero) length.
%
%    \begin{macrocode}
  \define@key{migl}{width}{%
    \setlength{\@tempdima}{#1}%
    \setlength{\@tempdimb}{\@tempdima / \real{62.091}}%
    \ifthenelse{%
      \lengthtest{\migl@unit@calc = 0pt}\OR
      \lengthtest{\@tempdimb < \migl@unit@calc}%
    }{%
      \setlength{\migl@unit@calc}{\@tempdimb}%
    }{}%
  }
%    \end{macrocode}
% \end{optionkey}
%
% \begin{optionkey}{scale}
% We also provide the option \key{scale} for unconditionally setting the scale to a
% given numeric factor, where 1 means \cs{migl@unit} equals 1pt.
%
%    \begin{macrocode}
  \define@key{migl}{scale}{%
    \setlength{\@tempdima}{1pt}%
    \setlength{\migl@unit@calc}{#1\@tempdima}%
  }
%    \end{macrocode}
% \end{optionkey}
%
% \begin{macro}{migl@line@color}
% \begin{optionkey}{outline}
% In case the logo is put on a coloured background, we provide the option
% \key{outline} for outlining it in a given colour (white by default).
%
%    \begin{macrocode}
  \newcommand{\migl@line@color}{none}
  \define@key{migl}{outline}[white]{\renewcommand{\migl@line@color}{#1}}

%    \end{macrocode}
% \end{optionkey}
% \end{macro}
%
% \begin{macro}{MIGLogo}
% The logo itself is drawn with the \cs{MIGLogo} command. The \pkg{keyval} options
% are read from the optional argument. There is no mandatory argument.
%
%    \begin{macrocode}
\newcommand{\MIGLogo}[1][]{%
  \bgroup
%    \end{macrocode}
%
% First we reset \cs{migl@unit@calc} to 0pt, then read in the user keys.
%
%    \begin{macrocode}
  \setkeys{migl}{scale=0,#1}%
%    \end{macrocode}
%
% If \cs{migl@unit@calc} has been set, we scale the logo accordingly. Otherwise
% we use the default scale of \cs{migl@unit} = 1pt.
%
%    \begin{macrocode}
  \ifthenelse{\lengthtest{\migl@unit@calc > 0pt}}{%
    \setlength{\migl@unit}{\migl@unit@calc}%
  }{%
    \setlength{\migl@unit}{1pt}%
  }%
%    \end{macrocode}
%
% Now we come to draw the logo. The \key{text} node is used to centre the logo
% vertically with respect to the surrounding text. The graphic itself is
% achieved as a set of Paisley swirls.
%
%    \begin{macrocode}
  \begin{tikzpicture}
  [ inner sep = 0pt
  , outer sep = 0pt
  , baseline = (text.base)
  , line width = 0.4pt
  ]
    \begin{scope}[yscale=0.833]
    \useasboundingbox (0,0) circle[radius=30\migl@unit];
    \foreach \angle/\color in {335/rdagray, 155/rdagreen, 65/rdabrown, 245/rdayellow} {
      \path[fill={\color}, draw={\migl@line@color}]
      ($(\angle:20.5\migl@unit) + (\angle + 190:9.5\migl@unit)$)
      arc[radius=9.5\migl@unit, start angle={\angle + 190}, end angle=\angle]
      arc[radius=30\migl@unit, start angle=\angle, end angle={\angle - 120}]
      to[out={\angle - 20}, in={\angle - 80}] cycle;
    }
    \end{scope}
    \node (text) at (0,0) {\phantom{RDA}};
  \end{tikzpicture}%
  \egroup
}

%    \end{macrocode}
% \end{macro}
%
% \subsubsection{Group information}
%
% \begin{macro}{rdaGroupLogo}
% \begin{macro}{rdaGroupName}
% \begin{macro}{rdaGroupType}
% \begin{macro}{rdaGroupUrl}
% Finally, we define hooks that are recognized by \texttt{rdaslides.cls}.
%
%    \begin{macrocode}
\def\rdaGroupLogo{\MIGLogo[width=\hsize]}
\def\rdaGroupName{Metadata}
\def\rdaGroupType{Interest Group}
\def\rdaGroupUrl{https://www.rd-alliance.org/groups/metadata-ig.html}
%    \end{macrocode}
% \end{macro}
% \end{macro}
% \end{macro}
% \end{macro}
%
% \iffalse
%</mig>
% \fi
%\Finale
